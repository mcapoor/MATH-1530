\documentclass[12pt]{article} 
\usepackage[utf8]{inputenc}
\usepackage{geometry}
\geometry{letterpaper}
\usepackage{graphicx} 
\usepackage{parskip}
\usepackage{booktabs}
\usepackage{array} 
\usepackage{paralist} 
\usepackage{verbatim}
\usepackage{subfig}
\usepackage{fancyhdr}
\usepackage{sectsty}
\usepackage{enumitem}

\pagestyle{fancy}
\renewcommand{\headrulewidth}{0pt} 
\lhead{}\chead{}\rhead{}
\lfoot{}\cfoot{\thepage}\rfoot{}


%%% ToC (table of contents) APPEARANCE
\usepackage[nottoc,notlof,notlot]{tocbibind} 
\usepackage[titles,subfigure]{tocloft}
\renewcommand{\cftsecfont}{\rmfamily\mdseries\upshape}
\renewcommand{\cftsecpagefont}{\rmfamily\mdseries\upshape} %

\usepackage{amsmath}
\usepackage{amssymb}
\usepackage{mathtools}
\usepackage{empheq}
\usepackage{xcolor}

\usepackage{tikz}
\usepackage{pgfplots}
\pgfplotsset{compat=1.18}

\newcommand{\ans}[1]{\boxed{\text{#1}}}
\newcommand{\vecs}[1]{\langle #1\rangle}
\renewcommand{\hat}[1]{\widehat{#1}}
\newcommand{\F}[1]{\mathcal{F}(#1)}
\renewcommand{\P}{\mathbb{P}}
\newcommand{\R}{\mathbb{R}}
\newcommand{\E}{\mathbb{E}}
\newcommand{\Z}{\mathbb{Z}}
\newcommand{\ind}{\mathbbm{1}}
\newcommand{\qed}{\quad \blacksquare}
\newcommand{\brak}[1]{\langle #1 \rangle}
\newcommand{\bra}[1]{\langle #1 |}
\newcommand{\ket}[1]{| #1 \rangle}

\title{Math 1530: Homework 4}
\author{Milan Capoor}
\date{17 October 2023}

\begin{document}
\maketitle
\section*{3.40}
    \emph{Let $R$ be a commutative ring.}
    \begin{enumerate}[label=(\alph*)]
        \item \emph{Let $c \in R$. Prove that }
        \[\{cr : r \in R\}\] 
        \emph{is an ideal of R. As noted in Definition 3.27, it is called the principal ideal generated by $c$ and is denoted by $cR$ or $(c)$.}

            \color{blue}
                An ideal is a subgroup under addition of $R$ which has the property 
                \[ar \in I \quad \forall a \in I, r \in R\]

                First, we seek to show that $cR = \{cr : r \in R\}$ a subgroup of $R$. Let $c_1, c_2$ be two elements in $R$. Then 
                \[c_1r + c_2r = (c_1 + c_2)r \quad \forall r \in R\]
                but since $R$ is a ring, $c_1 + c_2 \in R$ so $cR$ is closed under addition. Further, it has an identity because $1 \in R$ so 
                \[1r = r \in \{cr : r\in R\}\] 
                Similarly, since $R$ is a ring, it is already a subgroup under addition so every $r \in R$ has an inverse in $R$. Thus, 
                \[cr + cr^{-1} = c(r + r^{-1}) = c0 = 0\]
                so every element in $cR$ has an additive inverse. Finally, associativity is inherited from $R$. Thus, $cR$ is a subgroup of $R$. 

                Now to check the absorption property, take any element $a \in cR$. Clearly, it will have the form $a = cr_1$ for some $r_1 \in R$. Now we take another element $r_2 \in R$ and consider
                \[ar_2 = cr_1 r_2\]
                However, since $R$ is a ring, it is closed under multiplication so $r_1r_2 \in R$. Denote the product $r = r_1 r_2 \in R$. Then $ar_2 = cr$. So clearly it is a member of $cR$. Thus, $cR$ is an ideal of $R. \qed$
            \color{black}

        \item \emph{More generally, let $c_1,\dots, \, c_n \in R$. Prove that}
        \[\{r_1c_1 + r_2c_2 + \dots\, + r_nc_n : r_1, \dots,\, r_n \in R\}\]
        \emph{is an ideal of $R$. As noted in Example 3.29, it is called the ideal generated by $c_1, \dots,\, c_n$ and is denoted by $(c_1, \dots,\, c_n)$ or $c_1R + \dots\, + c_nR$}

            \color{blue}
                First we must show that $(c_1, \dots,\, c_n)$ is an additive subgroup of $R$:
                \begin{enumerate}
                    \item Additive closure is trivial from definition of sum of products of elements in $R$ 
                    \item Identity: $0 \in R$ so  
                    \[0 + \sum^n_{i=1} r_i c_i = \sum^n_{i=1} r_i c_i\] 
                    \item Inverses: $c_1, \dots,\, c_n$ are in $R$ so they have additive inverses ($-c_1, \dots,\, -c_n \in R$).
                    \[\sum_{i=1}^n r_ic_i + \sum_{i=1}^n r_i(-c_i) = \sum_{i=1}^n r_i(c_i - c_i) = 0\]
                    \item Associativity comes from $R$
                \end{enumerate}

                Now to show absorption, observe for $r, r_1, \dots,\, r_n \in R$
                \[r(\sum_{i=1}^n a_i r_i) = \sum_{i=1}^n a_i (rr_i)\]
                And $R$ is a ring so $rr_i \in R$ so 
                \[r(\sum_{i=1}^n a_i r_i) \in \{r_1c_1 + r_2c_2 + \dots\, + r_nc_n : r_1, \dots,\, r_n \in R\} \qed\]
            \color{black}
    \end{enumerate}
\pagebreak

\section*{3.41}
    \emph{Let $R$ be a commutative ring. Prove that $R$ is a field if and only if its only ideals are the zero ideal $(0)$ and the entire ring $R$}
        \color{blue}
            Suppose there is an ideal in $R$ where $I \neq (0)$. Consider an element $a \in I, a \neq 0$. If $R$ is a field, every (non-zero) element has an inverse so $b = a^{-1}$ exists with $b \in R$. Then, by absorption 
            \[ba = 1 \in I\]
            Now we take any element $r \in R$ and observe that because $1 \in I$ the absorber property also gives 
            \[r1 = r \in I\]
            therefore, $R = I$. 

            Now we want to show that other direction: if the only ideals are $(0)$ and $(1)$, then $R$ is a field. Let $R$ be a ring with those ideals. We pick any $a \in R, a \neq 0$ and consider $I = aR$. If $I \neq \{0\}$ then there is an element $a \in I$. Now we consider $a1 = a$ but $a \in I$ so $1 \in I$.
            
            Then for any $r \in R$,
            \[r1 = r \in I\]
            so $r = a^{-1}$ exists for any $a$. Thus, $R$ is a field. $\qed$
        \color{black}
\pagebreak

\section*{3.43}
    \emph{The goal of this exercise is to prove that every ideal in $\Z$ is a principal ideal.}
    \begin{enumerate}[label=(\alph*)]
        \item \emph{Let $I$ be a non-zero ideal in $\Z$. Prove that $I$ contains a positive integer.}
        
            \color{blue}
                Suppose that $I$ contains no positive integers. Since $I$ is non-zero, that means there is a negative integer in $I$. Choose a negative integer in $I$ and call it $a$. Then because $I$ is an ideal, we should be able to choose any element in $r \in \Z$ and the product $ar$ should be in $I$. Observe, however, that for any negative $r$, $ar$ is a positive integer. This is a contradiction so $I$ must contain a positive integer. $\qed$
            \color{black}
        
        \item \emph{Let $I$ be a non-zero ideal in $\Z$. Let $c$ be the smallest positive integer in $I$. Prove that every
        element of $I$ is a multiple of $c$. (Hint. Use division with remainder.)}

            \color{blue}
                Suppose $a$ is an element of $I$ which is not a multiple of $c$. 
                
                By the division algorithm, 
                \[a = cm + r\]
                for $m, r \in \Z$ and $0 < r < c$.

                Clearly $cm \in I$ by absorption so $r = a - cm \in I$ by closure. But since $0 < r < c$, we have a contradiction of the minimality of $c$. Therefore, every element in $I$ is a multiple of $c. \qed$
            \color{black}

        \item \emph{Prove that every ideal in $\Z$ is principal.}
        
            \color{blue}
                The zero ideal is trivially principal because its only element, $0$, is obviously a multiple of $0$. 

                By part (a), all non-zero ideals in $\Z$ contain a positive integer. By the ordering of the integers, they must 
                \begin{enumerate}
                    \item contain only one positive integer 
                    \item have a smallest positive integer. 
                \end{enumerate}
                If there is only one positive integer in $I$, then it is clearly the smallest positive integer in $I$. So every non-zero ideal in $\Z$ has a smallest positive integer. 

                By part (b), every element of $I$ is a multiple of that smallest positive integer $c$. Thus, every element in $I$ can be written $cr : r \in R$ which is precisely the definition of a principal ideal. Thus, every ideal in $\Z$ is principal. $\qed$
            \color{black}
    \end{enumerate}
\pagebreak

\section*{3.48}
    \emph{Let $R$ be a commutative ring and let $I$ and $J$ be ideals of $R$.}
    \begin{enumerate}[label=(\alph*)]
        \item \emph{Prove that the intersection $I \cap J$ is an ideal of $R$.}
        
            \color{blue}
                We need to verify that the intersection is an abelian group under addition and that it has the absorption property.

                To start, $I \cap J$ inherits associativity from $R$. 
                
                To see closure, let $x, y \in I \cap J$. Then (obviously) $x$ and $y$ are both members of each ideal. Clearly, $x + y \in I$ (because $I$ is an ideal and thus an abelian group under addition) and similarly $x + y \in J$ so $x + y \in I\cap J$. Thus the intersection is closed under addition. 

                Identity: Because $I$ and $J$ are ideals, $0 \in I$ and $0 \in J$ ($0 \in R$ because it is a ring, so with $x \in I \cup J$, $0x = 0$). As the additive identity is in both ideals, $0 \in I\cap J$.

                Inverse: Let $x \in I \cap J$. $I$ and $J$ are ideals so $x^{-1} \in I$ and $x^{-1} \in J$. Thus $x^{-1} \in I \cap J$

                Absorption: Let $x \in I \cap J$ and $r \in R$. Consider $rx$. Since $x \in I$, $rx \in I$ but additionally $x \in J$ so $rx \in J$. Hence, $rx \in I\cap J$. 
                
                As $I \cap J$ is an abelian subgroup under addition of $R$ and has the absorber property, it is an ideal of $R. \qed$
            \color{black}

        \item \emph{Prove that the ideal sum}
        \[I + J = {a + b : a \in I\text{ and }b \in J}\]
        \emph{is an ideal of $R$.}

            \color{blue}
                \begin{enumerate}
                    \item Associativity: comes from addition in $R$
                    
                    \item Closure: Let $a_1, a_2 \in I$ and $b_1, b_2 \in J$. Clearly $a_1 + b_1 \in I + J$ and $a_2 + b_2 \in I + J$. Then 
                    \begin{align*}
                        (a_1 + b_1) + (a_2 + b_2) &= a_1 + a_2 + b_1 + b_2 \qquad (\text{commutativity})\\
                        &= (a_1 + a_2) + (b_1 + b_2) \qquad (\text{associativity})\\
                        &= a' + b' \in I + J \qquad (a' \in I, b' \in J \text{ by closure})\\
                    \end{align*}
                    
                    \item Identity: Let $a \in I$, $b \in J$. Since $I$ and $J$ are ideals, $0_I$ (the identity for $I$) and $0_J$ (the identity for $J$) exist. Then $0_I + 0_J \in I + J$ and 
                    \[a + b + 0_I + 0_J = (a + 0_I) + (b + 0_J) = a + b \in I + J\]
                    
                    \item Inverse: Let $x \in I$ and $y \in J$. Because $I$ is an ideal, $x^{-1} \in I$ exists. Similarly, $y^{-1} \in J$ exists. Then $x + y \in I + J$ and $x^{-1} + y^{-1} \in I + J$ by definition of $I + J$. Consider, 
                    \begin{align*}
                        (x + y) + (x^{-1} + y^{-1}) &= x + x^{-1} + y + y^{-1} \qquad (\text{by commutativity})\\
                        &= 0 + 0 = 0 \in I + J
                    \end{align*}
                    
                    \item Absorption: Let $a + b \in I + J$ and $r \in R$. We seek to show that $r(a + b) \in I + J$. By distributivity, $r(a + b) = ra + rb$. As $a \in I$, $ra \in I$ and $rb \in J$ so $ra + rb \in I + J$
                \end{enumerate}
                As $I + J$ is an abelian subgroup under addition of $R$ and has the absorber property, it is an ideal of $R. \qed$
            \color{black}

        \item \emph{The ideal product of two ideals is defined to be}
        \[ IJ = \{a_1b_1 + a_2b_2 + \dots\, + a_nb_n : n \geq 1 \text{ and } a_1, \dots\, , a_n \in I \text{ and } b_1, \dots\, , b_n \in J\}.\]
        \emph{Prove that $IJ$ is an ideal of $R$}
    
            \color{blue}
                \begin{enumerate}
                    \item Associativity: comes from $R$
                    
                    \item Closure: Let 
                    \begin{gather*}
                        \{a_1, \dots, a_n: a_i \in I\}\\
                        \{b_1, \dots, b_n: b_i \in J\}\\ 
                        \{c_1, \dots, c_n: c_i \in I\}\\
                        \{d_1, \dots, d_n: d_i \in J\}
                    \end{gather*} 
                    so $\sum_{i=1}^n a_i b_i, \sum_{i=1}^n c_i d_i \in IJ$. Consider their sum:
                    \begin{align*}
                        \sum_{i=1}^n a_i b_i + \sum_{i=1}^n c_i d_i &= \sum_{i=1}^n a_i b_i + c_i d_i\\
                    \end{align*}
                    But since $a_i$ and $c_i$ are both in $I$ for all $1 \leq i \leq n$, we can define a new sequence 
                    \[a' = \{a_1, \dots, a_n, c_1, \dots, c_n\}\]
                    and similarly for $J$, 
                    \[b' = \{b_1, \dots, b_n, d_1, \dots, d_n\}\]
                    So, 
                    \[\sum_{i=1}^n a_i b_i + c_i d_i = \sum_{i=1}^{2n} a'_ib'_i \in IJ\]

                    \item Identity: $0_I \in I$ and $0_J \in J$ so for an element $\sum_{i=1}^n a_i b_i \in IJ$, $0_I 0_J \in IJ$ (with $n = 1$) and 
                    \[0_I 0_J + \sum_{i=1}^n a_i b_i = \sum_{i=1}^n a_i b_i\]
                    
                    \item Inverse: Let $\sum_{i=1}^n a_i b_i$ be an element in $IJ$. Then for each $\forall a_i \in I, \exists a_i^{-1} \in I$ and $\forall b_i \in J, \exists b_i^{-1} \in J$ because $I$ and $J$ are ideals. Thus, $\sum_{i=1}^n a_i^{-1} b_i^{-1} \in IJ$ so 
                    \begin{align*}
                        \sum_{i=1}^n a_i b_i + \sum_{i=1}^n a_i^{-1}b_i^{-1} &= \sum_{i=1}^n a_i b_i + a_i^{-1}b_i^{-1}\\
                        &= \sum_{i=1}^n a_i b_i + (b_ia_i)^{-1} \qquad (\text{by inverse properties})\\ 
                        &= \sum_{i=1}^n a_i b_i + (a_ib_i)^{-1} \qquad (\text{by commutativity})\\
                        &= 0
                    \end{align*}
                    
                    \item Absorption: Let $\sum_{i=1}^n a_i b_i$ be an element in $IJ$ and $r \in R$. Then 
                    \[r\left(\sum_{i=1}^n a_i b_i\right) = \sum_{i=1}^n ra_i b_i\]
                    But as $a_i \in I$ and $I$ ideal, $ra_i \in I \forall a_i$. Thus, 
                    \[\sum_{i=1}^n ra_i b_i = \sum_{i=1}^n a_i' b_i \in IJ\]

                \end{enumerate}
                As $IJ$ is an abelian subgroup under addition of $R$ and has the absorber property, it is an ideal of $R. \qed$

            \color{black}

        \item \emph{One might ask why the product $IJ$ of ideals isn't simply defined as the set of products} 
        \[{ab : a \in I \text{ and } b \in J}\] 
        \emph{The answer is that the set of products need not be an ideal. Here is an example. Let $R = \Z[x]$, and let $I$ and $J$ be the ideals}
        \[I = 2\Z[x] + x\Z[x] \text{ and } J = 3\Z[x] + x\Z[x].\]
        \emph{Prove that the set of products $\{ab : a \in I \text{ and } b \in J\}$ is not an ideal.}

            \color{blue}
                $I$ can also be expressed as the set of polynomials 
                \[a \in I = 2a_0 + \sum_{i=1}^n a_ix^i\] 
                Similarly, $J$ is  
                \[b\in J = 3b_0 + \sum_{i=1}^n b_ix^i\]
                
                Thus to show that the set of products of $ab$ is not an ideal, we need to show that it is not closed under addition. 

                Consider $6 + x$. This is not itself a product $ab$ of elements in $I$ and $J$ because if both $a$ and $b$ are of degree 1, then $ab$ is of degree two. Similarly, if both $a$ and $b$ are of degree $0$ then $ab$ is of degree 0. Finally, assume WLOG that $a$ is of degree $0$ and $b$ is of degree 1. Then $ab = a(b_1 + x) = ab_1 + ax \implies a = 1$ but $1 \not \in I$. Hence we have a contradiction so $6 = x$ is not a product. 

                But 
                \[6 + x = 3(4 + x) - 2(3 + x)= 12 +3x - 6 - 2x\]
                and clearly $3(4 + x), -2(3 + x) \in \{ab\}$ because $3, 3 + x \in J$, and $4 + x, -2\in I$ so we have shown that there is an element which the sum of two elements in $\{ab\}$ which is not itself in $\{ab\}$. Therefore, $\{ab\}$ is not closed under addition so it is not an ideal. $\qed$                
            \color{black}

        \item \emph{On the other hand, prove in general that if either I or J is a principal ideal, then the set of products $\{ab : a \in I \text{ and } b \in J\}$ is an ideal.}
        
            \color{blue}
                Denote $I \circ J = \{ab: a \in I \text{ and } b \in J\}$
                \begin{enumerate}
                    \item Associativity: comes from $R$ 
                    
                    \item Closure: As either $I$ or $J$ is principal, assume WLOG it is $I$ that is principal. Then let $a_1 = ca_1', a_2 = ca_2'$ be any two elements in $I$. Further let $b_1, b_2 \in J$. Then for the arbitrary composition of two elements in $I \circ J$, 
                    \[a_1b_1 + a_2b_2 = ca_1'b_1 + ca_2'b_2 = c(a_1'b_1 + a_2'b_2)\]
                    As $I$ and $J$ are ideals of $R$, $a_1'b_1 + a_2'b_2 \in R$ so $c(a_1'b_1 + a_2'b_2) \in I \circ J$ 

                    \item Identity: As both $I$ and $J$ are ideals, $0 \in I \cap J$ because $0\in R$ so $\forall a \in I: 0a \in I$ and $\forall b \in J: 0b \in J$ so both $0b = 0$ and $a0 = 0$ are in $I\circ J$. Therefore, $ab + 0 \in I\circ J$.

                    \item Inverse: $\forall a \in I, \; \exists a^{-1} \in I$ and $\forall b \in J,\; \exists b^{-1} \in J$. So $a^{-1}b^{-1} \in I\circ J$. Further, by commutativity, $a^{-1}b^{-1} = b^{-1}a^{-1} = (ab)^{-1}$ so the inverse
                    \[ab + (ab)^{-1} = 0\] 
                    exists in $I\circ J$ for all $a, b$

                    \item Absorption: Let $ab \in I\circ J $, $r \in R$. By associativity $r(ab) = (ra)b$ and by $a \in I$ with $I$ ideal, $ra \in I$ so $(ra)b \in I\circ J$
                \end{enumerate}
                As $I \circ J$ is an abelian subgroup under addition of $R$ and has the absorber property, it is an ideal of $R. \qed$
            \color{black}
    \end{enumerate}

\pagebreak

\section*{3.52}
    \begin{enumerate}[label=(\alph*)]
        \item Let $m \neq 0$ be an integer. Prove that the ideal $m\Z$ is a prime ideal (and hence also a maximal ideal) if and only if $|m|$ is a prime number in the usual sense of primes in $\Z$.
        
            \color{blue}
                We need to show both that $m\Z$ being a prime ideal implies $\big\vert m \big\vert$ is prime and that $\big\vert m \big\vert$ being prime implies $m\Z$ is a prime ideal. 

                Beginning with second statement, we observe that if $\big\vert m \big\vert$ is prime, then by Proposition 3.20 $\Z/m\Z$ is a field. By Theorem 3.43, $R/I$ is a field if and only if $I$ is maximal, so $m\Z$ is maximal. Corollary 3.44 neatly completes the proof by showing that every maximal ideal is a prime ideal. 

                To see the opposite direction, note that $m\Z$ is a prime ideal if and only if $\Z/m\Z$ is an integral domain. Thus, $\Z/m\Z$ has no zero divisors. Suppose $\big\vert m \big\vert$ is composite, i.e. $ab = m$ for some $a, b \in \Z$. Then
                \[(a + m\Z)(b + m\Z) = ab + m\Z = 0 + m\Z\]
                but this is a contradiction of the fact that $\Z/m\Z$ is an integral domain. Hence, $\big\vert m \big\vert$ must be prime. $\qed$
            \color{black}

        \item Let $F$ be a field, and let $a, b \in F$ with $a\neq 0$. Prove that the principal ideal $(ax + b)F[x]$ is a maximal ideal of the polynomial ring $F[x]$.

            \color{blue}
                The ideal $I = (ax + b)F[x]$ is a maximal ideal of $R = F[x]$ if and only if $R/I$ is a field. That is, with $p(x), q(x) \in F[x]$,
                \[(p(x) + I)(q(x) + I) = p(x)q(x) + I = 0 + I\]
                So, equivalently, we want to show that $p(x)q(x) \in (ax + b)F[x]$ 

                But as $I$ is principal, every element of $I$ is divisible by $ax + b$. So we need to show that $ax + b \Big\vert p(x)q(x)$ for all $p, q \in F[x]$. 
                
                We are concerned with the product $p(x)q(x)$ in a ring of polynomials so the product will also be in $F[x]$. For simplicity we will let $f(x) = p(x)q(x)$. Then we apply polynomial division:
                \[f(x) = (ax + b)m + r\]
                where $m, r\in F[x]$ and $0 \leq \deg(r) < \deg(ax + b) = 1$. Thus we know $f(x)$ is in a constant coset $f(x) - r \in I$. So by closure, $p(x)q(x) \in I. \qed$
            \color{black}
    \end{enumerate}
\pagebreak

\section*{3.54}
    Let $R$ be a ring, let $b, c \in R$, and let $E_{b,c} : R[x, y] \to R$ be the evaluation homomorphism described in Exercise 3.13.

    (Hint. Use Proposition 3.34 and Theorem 3.43.)
    \begin{enumerate}[label=(\alph*)]
        \item If $R$ is an integral domain, prove that $\ker(E_{b,c})$ is a prime ideal of $R[x, y]$.
        
            \color{blue}
                By proposition 3.34b(i), the kernel of a ring homomorphism $\phi: R \to R'$ is an ideal of $R$, so we know that $\ker(E_{b, c})$ is an ideal of $R[x, y]$. 

                Now we need to show that it is prime. From Theorem 3.43a, $I$ is a prime ideal if and only if the quotient ring $R/I$ is an integral domain. So in this case, we need to show that $R[x,y]/\ker(E_{b, c})$ is an integral domain. 

                Applying 3.34b(iii), we then know there is an injective ring homomorphism from $R[x,y]/\ker(E_{b, c}) \to R$. In fact, if we can show that $E_{b,c} : R[x,y] \to R$ is surjective, then by the isomorphism theorem, the map $R[x,y]/\ker(E_{b, c}) \to R$ is actually an isomorphism. 

                From the definition of $E_{b,c}$,
                \[E_{b,c}[f(x, y)] = f(b, c) = \sum_{i=0}^m \sum_{j=0}^n a_{ij}b^ic^j\]
                This is clearly surjective because with $a_{ij} \in R$, 
                \[a_{ij} = a_{ij}b^0c^0 + \sum_{i=1}^m \sum_{j=1}^n 0b^ic^j\]
                
                Therefore, we have a surjective homomorphism from $R[x, y] \to R$, a homomorphism from $R[x, y] \to R/\ker(E_{b,c})$, and an isomorphism from $R[x,y]/\ker(E_{b, c}) \to R$.

                Finally, if $R$ is an integral domain and there is an isomorphism from $R[x, y]/\ker(E_{b,c})$ to $R$, $R[x, y]/\ker(E_{b,c})$ must also be an integral domain because isomorphic rings have the same structure. Thus from theorem 3.43, $\ker(E_{b,c})$ is a prime ideal. $\qed$
            \color{black}
        \pagebreak
        \item If $R$ is a field, prove that $\ker(E_{b,c})$ is a maximal ideal of $R[x, y]$.
        
            \color{blue}
                From above, we know that $\ker(E_{b,c})$ is an ideal of $R[x, y]$. And by Theorem 3.43, an ideal $I$ is maximal if and only if the quotient ring $R/I$ is a field. 

                We already showed that $R[x,y]/\ker(E_{b, c})$ is isomorphic to $R$ so if $R$ is a field, then $R[x,y]/\ker(E_{b, c})$ is a field so $\ker(E_{b, c})$ is maximal. $\qed$ 
            \color{black}

    \end{enumerate}

\end{document}