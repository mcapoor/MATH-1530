\documentclass[12pt]{article} 
\usepackage[utf8]{inputenc}
\usepackage{geometry}
\geometry{letterpaper}
\usepackage{graphicx} 
\usepackage{parskip}
\usepackage{booktabs}
\usepackage{array} 
\usepackage{paralist} 
\usepackage{verbatim}
\usepackage{subfig}
\usepackage{fancyhdr}
\usepackage{sectsty}
\usepackage{enumitem}

\pagestyle{fancy}
\renewcommand{\headrulewidth}{0pt} 
\lhead{}\chead{}\rhead{}
\lfoot{}\cfoot{\thepage}\rfoot{}


%%% ToC (table of contents) APPEARANCE
\usepackage[nottoc,notlof,notlot]{tocbibind} 
\usepackage[titles,subfigure]{tocloft}
\renewcommand{\cftsecfont}{\rmfamily\mdseries\upshape}
\renewcommand{\cftsecpagefont}{\rmfamily\mdseries\upshape} %

\usepackage{amsmath}
\usepackage{amssymb}
\usepackage{mathtools}
\usepackage{empheq}
\usepackage{xcolor}

\usepackage{tikz}
\usepackage{pgfplots}
\pgfplotsset{compat=1.18}

\renewcommand{\hat}[1]{\widehat{#1}}
\newcommand{\F}[1]{\mathcal{F}(#1)}
\renewcommand{\P}{\mathbb{P}}
\newcommand{\R}{\mathbb{R}}
\newcommand{\E}{\mathbb{E}}
\newcommand{\Z}{\mathbb{Z}}
\newcommand{\ind}{\mathbbm{1}}
\newcommand{\qed}{\quad \blacksquare}
\newcommand{\brak}[1]{\left\langle #1 \right\rangle}
\newcommand{\bra}[1]{\left\langle #1 \right\vert}
\newcommand{\ket}[1]{\left\vert #1 \right\rangle}
\newcommand{\Q}{\mathbb{Q}}

\title{Math 1530: Homework 6}
\author{Milan Capoor}
\date{31 October 2023}

\begin{document}
\maketitle
\section*{5.9}
    \begin{enumerate}[label=(\alph*)]
        \item \emph{Let $K / F$ be an extension of fields. Prove that}
        \[[K : F] = 1 \iff K = F.\]

            \color{blue}    
                If $K = F$, then for all $k \in K$,
                \[k = f \quad \text{for some } f \in F\]
                written differently, 
                \[k= f\cdot 1\]
                so $\{1\}$ is a spanning set for $K$ over the $F$-vector space. But additionally, $\{1\}$ is linearly independent because $a1 = 0 \implies a = 0$. Thus, $\{1\}$ is a basis for $F/K$ and 
                \[\dim_F(K) = \#\{1\} [K : F] = 1\]

                For the other direction, we suppose that $[K : F] = 1$. 
                By definition, $[K : F] = \dim_F(K) = 1$. So the basis for $K$ consists of only one vector. Put differently, every $k \in K$ is a multiple of some element in $F$ and the basis vector $v \in K$:
                \[K = \{fv : f\in F\}\]
                But since $K / F$ is an extension of fields, $F \subset K$ so $f \in K \implies v = 1$. But this means that 
                \[K = \{f1 : f\in F\} = \{f: f \in F\} = F \qed\]
            \color{black}

        \item \emph{Let $L/ F$ be a finite extension of fields, and suppose that $[L : F]$ is prime. Suppose further that $K$ is a field lying between $F$ and $L$; i.e., $F \subseteq K \subseteq L$. Prove that either $K = F$ or $K=L$. }
        
            \color{blue}
                Assume that $K \neq F$ and $K \neq L$. Then because $F \subseteq K \subseteq L$, $K$ creates a tower of field extensions $L/K/F$. 

                By Theorem 5.18,
                \[[L : F] = [L : K] \cdot [K : F]\]
                but since $[L:F]$ is prime, we have two cases:
                \begin{enumerate}
                    \item $[L:K] = [L:F], \quad [K:F] = 1$
                    \item $[K:F] = [L:F], \quad [L:K] = 1$
                \end{enumerate}

                From Part (a), if $[K:F] = 1$, $K = F$ and if $[L:K] = 1$, $L = K$ so we have a contradiction and $K \neq F$ or $K \neq L. \qed$
            \color{black}
    \end{enumerate}
\pagebreak

\section*{5.14}
   \emph{ This exercise asks you to prove the uniqueness of the quotient and remainder appearing in Proposition 5.20. Let $F$ be a field, let $f(x),g(x) \in F(x)$ be polynomials with $g(x) \neq 0$, and suppose that there are polynomials $q_1(x), q_2(x), r_1(x), r_2(x) \in F(x)$ satisfying}
    \begin{align*}
        f(x) = g(x)q_1(x) + r_1(x) \quad \text{with } \deg(r_1) < \deg(g),\\
        f(x) = g(x)q_2(x) + r_2(x) \quad \text{with } \deg(r_2) < \deg(g).
    \end{align*}
   \emph{Prove that $q_1(x) = q_2(x)$ and $r_1(x) = r_2(x)$.}

   \color{blue}
        \begin{align*}
            g(x)q_1(x) + r_1(x) = g(x)q_2(x) + r_2(x)\\
            g(x)q_1(x) - g(x)q_2(x) = r_2(x) - r_1(x)\\
            g(x)[q_1(x) - q_2(x)] = r_2(x) - r_1(x)
        \end{align*}
        But $\deg(r_2(x) - r_1(x)) \leq \deg(r_2) < \deg(g)$ so the only way for the LHS and RHS to be equal is if $q_1(x) - q_2(x) = 0 \implies q_1(x) = q_2(x)$. Then, since the LHS is equal, $r_2(x) - r_1(x) = 0 \implies r_2(x) = r_1(x). \qed$
   \color{black}
\pagebreak

\section*{5.15}
    \emph{Let $F$ be a field.}
    \begin{enumerate}[label=(\alph*)]
        \item \emph{Prove that every polynomial of degree $1$ in $F[x]$ is irreducible.}
        
            \color{blue}
                Let $ax + b$ ($a \neq 0$) be any polynomial of degree 1 in $F[x]$. Consider the ideal $(ax + b)F[x]$. Suppose that there is an ideal $I$ such that 
                \[(ax + b)F[x] \subseteq I \subseteq F[x]\]

                From Theorem 5.21, every ideal in $F[x]$ is principal so $\exists g(x)\in F[x]: I = g(x)F[x]$, so
                \[(ax + b)F[x] \subseteq g(x)F[x] \subseteq F[x]\]
                But this means that $ax + b \in (g(x))$ so 
                \[ax + b = g(x)h(x) \quad h(x) \in F[x]\]

                But $\deg(gh) = \deg(g) + \deg(h)$ (Definition 5.19) and $\deg(ax + b) = 1$ so 
                \[\deg(g) + \deg(h) = 1\] 
                
                Further, $\deg(g) \geq 1$ or else $ax + b \not \in g(x)F[x]$ so $\deg(h) = 0$. Thus $h(x) = c \in F$. So $ax + b = cg(x)$ and $\deg(g) = 1$. So $ax + b = g(x)$ up to a constants and the only factorization of $ax + b$ is trivial. Thus, any polynomial of degree $1$ is irreducible. $\qed$

            \color{black}

        \item \emph{Let $f(x) \in F[x]$ be a polynomial of degree $2$. Prove that $f(x)$ is irreducible if and only if it has no roots in $F$.}
        
            \color{blue}
                Assume that $f$ has no roots in $f$. Then it cannot be written in the form 
                \[(ax + b)(cx + d)\]
                with $a, b, c, d \in F$ because then $x = -\frac{b}{a}$ and $x =- \frac{d}{c}$ would be roots. But this means that there are no degree 1 factorizations of $f$, so it is irreducible. 

                For the other direction, we want to show that if it is irreducible, it has no roots in $F$. By contraposition, if $f$ is reducible, it \emph{does} have roots in $F$. This is much easier because if $f$ is reducible, then it can be written as the product of two linear polynomials because it itself is quadratic:
                \[f(x) = (ax + b)(cx + d)\]
                But this polynomial has roots at $-\frac{b}{a}$ and $-\frac{d}{c}$ which are in $F$ by existence of inverses and closure of $F$. Thus, if $f$ is reducible, it has roots in $F$. So we conclude that $f$ is irreducible if and only if it has no roots in $F$. $\qed$
                
            \color{black}

        \item \emph{Let $f(x) \in F[x]$ be a polynomial of degree $3$. Prove that $f(x)$ is irreducible if and only if it has no roots in $F$.}
        
            \color{blue}
                In a very similar approach to the above, we will seek to prove the easier contrapositive: $f$ is reducible iff it has roots in $F$. 

                If $f$ is reducible, it can be written in the form 
                \[f(x) = (ax + b)(cx^2 +dx + e)\] 
                because $\deg(ax + b) + \deg(cx^2  +dx + e) = 3$. But, then $x = -\frac{b}{a} \in F$ would be a root because 
                \[f(-\frac{b}{a}) = (a\cdot (-\frac{b}{a}) + b)(c(-\frac{b}{a})^2 +d(-\frac{b}{a}) + e) = 0 \cdot (c(-\frac{b}{a})^2 +d(-\frac{b}{a}) + e) = 0\]

                Meanwhile, if $f$ has a root $a$ in $F$ then $(x - a)$ is a factor of $f$ by the splitting field. Then, having a degree 1 factor, $f$ is reducible. 

                Thus $f$ is reducible iff it has roots in $F$ and by the Law of Contraposition, $f(x)$ is irreducible if and only if it has no roots in $F$. $\qed$
            \color{black}

        
        \item \emph{Let $f(x) = x^4 + 2$. Prove that $f(x)$ is irreducible in $\Q[x]$.}
        
            \color{blue}
                If $f$ is reducible, then it can be written in the form 
                \[f(x) = (ax^2 + bx + c)(dx^2 + ex + f) = (ad)x^4 + (ae + bd)x^3 + (af + be + cd)x^2 + (ce + bf)x + cf\]
                But we want 
                \[\begin{cases}
                    ad = 1\\
                    ae + bd = 0\\
                    af + be + cd = 0\\
                    ce + bf = 0\\
                    cf = 2
                \end{cases}\]
                so from the middle three equations, 
                \[\begin{pmatrix}
                    e & d & 0\\
                    f & e & d\\
                    0 & f & e
                \end{pmatrix} \begin{pmatrix}
                    a\\b\\c
                \end{pmatrix} = \begin{pmatrix}
                    0\\0\\0
                \end{pmatrix}\]
                which gives the solution 
                \[\begin{cases}
                    a = 0\\
                    b= 0\\
                    c = 0
                \end{cases}\]
                
                But this leads to contradictions with 
                \[(0)d = 1\]
                and 
                \[(0)f = 2\]
                so $f$ must be irreducible. $\qed$

            \color{black}

        \item \emph{Let $f(x) = x^4 + 4$. Prove that $f(x)$ is reducible in $\Q[x]$, despite the fact that it has no roots in $\Q$. }
        
            \color{blue}
                If $f$ is reducible, it can be written $f(x) = p(x)q(x)$ where $\deg(p) + \deg(q) = 4$. The natural first guess is that both $p$ and $q$ are polynomials of degree $2$. Further, we want them both to be monic or to have their leading coefficients be multiplicative inverses. Finally, we want the product of their constant terms to be $4$. We will make a first guess of
                \begin{align*}
                    (x^2 + ax + 2)(x^2 + bx + 2) &= x^4 + ax^3 + 2x^2 + bx^3 + abx^2 + 2ax + 2x^2 + 2bx + 4\\
                    &= x^4 + ax^3 + bx^3 + 4x^2 + abx^2 + 2ax + 2bx + 4\\
                    &= x^4 + (a + b)x^3 + (4 + ab)x^2 + (2a + 2b)x + 4
                \end{align*}
                Setting this equal to $x^4 + 4$ gives us a system of equations constraining $a$ and $b$:
                \[\begin{cases}
                    a + b = 0\\
                    4 + ab = 0\\
                    2a + 2b = 0
                \end{cases} \implies \begin{cases}
                    a = 2\\
                    b = -2
                \end{cases}\]

                So we have found a factorization 
                \[x^4 + 4 = (x^2 + 2x + 2)(x^2 -2x + 2)\]
                and thus $f$ is reducible in $\Q[x]. \qed$
            \color{black}
    \end{enumerate}


\pagebreak

\section*{5.20}
\emph{Let F be a field, let $f(x) \in F[x]$ be a possibly reducible non-constant polynomial, and let $d = deg(f)$.}
\begin{enumerate}[label=(\alph*)]
    \item \emph{Prove that there exists a field extension $K / F$ satisfying $[K : F] \leq d$ such that $f(x)$ has a root in $K$.}
    
        \color{blue}
            If $f$ is irreducible, then by Theorem 5.27, $K = F[x]/f(x)F[x]$ is a finite extension field of $F$ with degree $[K: F] = \deg(f) = d$ which contains a root of $f(x)$. 

            If $f$ is reducible, then we can say $f(x) = g(x)h(x)$ with $\deg(g) < d$ and $\deg(h) < d$. If either $g$ or $h$ is irreducible, then by the case above, the extension field $F[x]/g(x)F[x]$ (or $F[x]/h(x)F[x]$) contains a root of $g$ (or $h$). Then since that polynomial is a factor of $f$, the root is also a root of $f$ and the degree of the extension field will be less than $d$. 

            If neither $g$ nor $h$ is irreducible, then each can be written as the product of two more polynomials of lesser degree. Again, these polynomials will either be irreducible and contain a root of $f$ or we can find yet smaller factors of $f$. Eventually, $f$ will have an irreducible polynomial as a factor because all degree 1 polynomials are irreducible (5.15a). 

            Thus, any $f(x) \in F[x]$ has a root in some extension field of degree less than or equal to $\deg(f). \qed$
        \color{black}
    
    \item \emph{Prove that there exists a field extension $L/ F$ and elements $c \in F$ and $a_1, \dots,\; a_d \in L$ such that $f(x)$ factors in $L[x]$ as}
    \[f(x) = c(x - a_1)(x - a_2) \dots (x - a_d)\]
    \emph{ (Note that $a_1, \dots,\; a_d$ need not be distinct.) Prove that it is always possible to find such an $L$ that also satisfies}
    \[[L : F] \leq d!\]
    \emph{The field $L$ is called a splitting field for the polynomial $f(x)$ over the field $F$}

        \color{blue}
            If $\deg(f) = d = 1$, then $f(x) = ax + b$ and we can factor out the leading coefficient such that it has a monic linear polynomial as a factor. 

            For $d \geq 1$, we note that from part (a), we have an extension $G/F$ where $G = F[x]/g(x)F[x]$ which contains a root of $g(x)$, one of the factors of $f$. We call this root $a_1 \in G$ and note that $g$ must be of the form $g(x) = x - a_1$ because it is irreducible. Thus, 
            \[f(x) = c(x - a_1)h(x)\]
            where $c \in F$, $h(x) \in F[x]$ and $\deg h = d - 1$. 

            Our approach to finding $g$ and the root $a_1$ in part (a) depended on a tower of reducible polynomials factoring into irreducible polynomials:
            \[f(x) = g(x) \cdot \prod_{i} h_i(x)\]
            where each $h_i(x)$ is an irreducible polynomial of the form 
            \[h_i(x) = x - a_i\] 
            and each $a_i$ is a root of $h_i$ in the extension field $H_i = F[x]/h_i(x)F[x]$.

            By proposition 5.15, the smallest field that contains $F[x]$ and all the roots $a_1, \dots, \; a_d$ is $L = F(a_1, a_2, \dots,\; a_d)$ so $L/F$ is a field extension such that $f$ factors in $L[x]$ as 
            \[f(x) = c(x - a_1)(x - a_2) \dots (x - a_d)\]
            ($c \in F$). 

            Now consider the degree of this field. In part (a), we found the first root in 
            $[G:F] \leq d$. But the order of roots does not matter because of commutativity in the polynomial ring so equivalently, any root can be found in an extension field of degree at most $d$. But again by Proposition 5.15, to have two roots (say in extensions $H_1$ and $H_2$), we need to construct the larger extension $H_2/H_1/F$.

            By theorem 5.18, 
            \[[H_2 : F] = [H_2 : H_1][H_1 : F] \leq (d-1)\cdot d\]
            so, generalizing, 
            \[[L :F] = [L:H_d][H_d : H_{d-1}]\dots[H_1:F] \leq d \cdot (d-1)\cdot (d-2)\cdot 2\cdot 1 = d! \qed\]
        \color{black}
\end{enumerate}
\pagebreak

\section*{5.21ab}
    \emph{Let $F$ be a finite field with $q$ elements.}

    \begin{enumerate}[label=(\alph*)]
        \item \emph{Prove that every non-zero element of $F$ is a root of the polynomial $x^{q-1} - 1$. (Hint. Apply the corollary of Lagrange's Theorem, Corollary 2.50, to the group of units $F^*$ .)}
        
            \color{blue}
                Because $F$ is a field, the set of non-zero elements of $F$ is the group of units $F^*$. By Corollary 2.50, the order of every element of this group must divide the order of the group, $q - 1$. i.e., for any $x \in F^*$
                \[o(x) \cdot m = q - 1 \quad (m \in \Z)\] 
                so
                \[x^{q - 1} = x^{o(x) \cdot m} = (x^{o(x)})^m = 1^m = 1\]

                This means that every element in the group of units of $F$ satisfies the equation
                \[x^{q-1} - 1 = 0\]
                so every non-zero element of $F$ is a root of $x^{q-1} - 1. \qed$
            \color{black}

        \item \emph{Prove that every element of $F$ is a root of the polynomial $x^q - x$.} 
        
            \color{blue}
                Trivially, $x = 0$ is a root. 

                Now we consider the non-zero elements. By part (a), every non-zero element $x \in F$ satisfies $x^{q-1} - 1 = 0$. Because $F$ is a field, multiplication is closed so we can take the product 
                \[x(x^{q-1}) - x = 0 \implies x^{q} - x = 0 \quad \forall x \in F^*\]
                so every $x \in F^*$ is a root of $x^q - x$. 

                Since every non-zero element and the zero element of $f$ are roots of $x^q - q$, every element of $F$ is a root. $\qed$
            \color{black}
    \end{enumerate}
\end{document}