\documentclass[12pt]{article} 
\usepackage[utf8]{inputenc}
\usepackage{geometry}
\geometry{letterpaper}
\usepackage{graphicx} 
\usepackage{parskip}
\usepackage{booktabs}
\usepackage{array} 
\usepackage{paralist} 
\usepackage{verbatim}
\usepackage{subfig}
\usepackage{fancyhdr}
\usepackage{sectsty}
\usepackage{enumitem}
\usepackage{multicol}


\pagestyle{fancy}
\renewcommand{\headrulewidth}{0pt} 
\lhead{}\chead{}\rhead{}
\lfoot{}\cfoot{\thepage}\rfoot{}


%%% ToC (table of contents) APPEARANCE
\usepackage[nottoc,notlof,notlot]{tocbibind} 
\usepackage[titles,subfigure]{tocloft}
\renewcommand{\cftsecfont}{\rmfamily\mdseries\upshape}
\renewcommand{\cftsecpagefont}{\rmfamily\mdseries\upshape} %

\usepackage{amsmath}
\usepackage{amssymb}
\usepackage{mathtools}
\usepackage{empheq}
\usepackage{xcolor}

\usepackage{tikz}
\usepackage{pgfplots}
\pgfplotsset{compat=1.18}

\newcommand{\ans}[1]{\boxed{\text{\#1}}}
\renewcommand{\hat}[1]{\widehat{\#1}}
\newcommand{\F}[1]{\mathcal{F}(\#1)}
\renewcommand{\P}{\mathbb{P}}
\newcommand{\R}{\mathbb{R}}
\newcommand{\E}{\mathbb{E}}
\newcommand{\Z}{\mathbb{Z}}
\newcommand{\ind}{\mathbbm{1}}
\newcommand{\qed}{\quad \blacksquare}
\newcommand{\brak}[1]{\left\langle #1 \right\rangle}
\newcommand{\bra}[1]{\left\langle #1 \right\vert}
\newcommand{\ket}[1]{\left\vert #1 \right\rangle}
\renewcommand{\mod}{\;\text{mod}\;}

\setenumerate[0]{label=(\alph*)}

\title{Math 1530: Homework 8}
\author{Milan Capoor}
\date{21 November 2023}

\begin{document}
\maketitle

\section*{6.13} 
Let $G$ be a group, and let $X$ be a set on which $G$ acts.
\begin{enumerate}
    \item Suppose that $\#G = 15$ and $\#X = 7.$ Prove that there is some element of $X$ that is fixed by every element of $G$. (Hint. Use the Orbit-Stabilizer Counting Theorem.)
    
        \color{blue}
            We want to show that there exists an $x \in X$ such that $gx = x$ for all $g \in G$. Equivalently, $G_x = G$ for some $X \in X$. By the Orbit-Stabilizer Counting Theorem, 
            \[\#X = \sum_{i=1}^k \#Gx_i = \sum_{i=1}^k \frac{\#G}{\# G_{x_i}} \implies 7 = \sum_{i=1}^k \#Gx_i = \sum_{i=1}^k \frac{15}{\#G_{x_i}}\]
            
            Since $\#G_x$ divides $\#G$ by Lagrange's theorem, we know that $\#G_{x_i} = \{1, 3, 5, 15\}$. But since $\#G$ is finite, 
            \[\#Gx_i = \frac{\#G}{\#G_{x_i}}\]
            so the quotient must be an integer not greater than $7$. Thus, each $\#G_{x_i} = \{3, 5, 15\}$ and $\#Gx_i = \{1, 3, 5\}$
            
            However, since $\#X = 7$, we must have that $\#Gx_i = 1$ for some $i$ because $3 \not \mid 7$ and $5 \not \mid 7$. Thus, $\#G_{x_i} = 15$ and $G_{x_i} = G$ for some $x_i \in X$ and there is some element fixed by every element of $G. \qed$
           
        \color{black}

    \item What goes wrong with your proof in (a.) if $\#G = 15$ and $\#X = 6$ or $\#X = 8$?

        \color{blue}
            If $\#X = 6$, then there is a possibility there is no orbit of size 1 (and thus no stabilizer of size 15) because 
            \[6 = 3+3\] 
            and $3 \in \{\frac{\#G}{\#G_{x_i}}\}$. Similarly, if $\#X = 8$, 
            \[8 = 3 + 5\]
            and $5 \in \{\frac{\#G}{\#G_{x_i}}\}.$ Thus, there may not be an element fixed by every element of $G. \qed$
        \color{black}

\end{enumerate}

\pagebreak
\section*{6.15abc} 
Let $G$ act on itself by conjugation as described in the proof of Theorem 6.25. Let $x \in G$. The conjugacy class of $x$ is the orbit of $x$ for this action; i.e., it is the set $\{gxg^{-1} : g \in G\}$.
\begin{enumerate}
    \item Prove that $G$ is the disjoint union of its conjugacy classes.
    
        \color{blue}
            By Proposition 6.19(b), there is an equivalence relation on $X$ such that the equivalence class of $x$ is its orbit $Gx$. By Theorem 1.25b, $X$ is the disjoint union of the distinct equivalence classes so
            \[X = Gx_1 \cup Gx_2 \cup \dots \cup Gx_k\]
            However, here with $X = G$, $Gx_i$ is just the conjugacy class of $x_i$ so $G$ is the disjoint union of its conjugacy classes. $\qed$
        \color{black}

    \item Prove that $G$ is abelian if and only if each conjugacy classes of $G$ contains a single element.

        \color{blue}
            If $G$ is abelian, then for all $x \in G$, $gxg^{-1} = xgg^{-1} = x$ for all $g \in G$. Thus, the conjugacy class of $x$ is $\{x\}$ and all conjugacy classes have only a single element. 

            For the other direction, suppose each conjugacy class contains only a single element, say $x'$, such that 
            \[\{x'\} = \{gxg^{-1} : g \in G\} \implies \#Gx = 1\]
            
            By Proposition 6.19c, 
            \[\#Gx = \frac{\#G}{\#G_x} \implies \#G_x = \#G \implies G_x = G\]

            The stabilizer of $x$ is  
            \[G_x = \{gxg^{-1} = x: g \in G\} = \{gx = xg: g \in G\} = G\]
            so $G$ is abelian. $\qed$
        \color{black}

    \item Describe the conjugacy classes of the dihedral group $D_3$.
    
        \color{blue}
            $D_3 = \{e, r_1, r_2, f_1, f_2, f_3\}$. We will consider the conjugacy classes of each element in turn using the composition formulae in Example 2.46.
            \begin{multicols}{2}
                \begin{enumerate}
                    \item The conjugacy class of $e$ is $\{e\}$ because $geg^{-1} = gg^{-1} = e$ for all $g \in G$. By part (a), we know that the conjugacy classes of $D_3$ are the disjoint union of the conjugacy class of $e$ and the conjugacy classes of the other elements. Thus, the other conjugacy classes of $D_3$ do not contain $e$. 
                    
                    \item $r_1$:
                        \begin{align*}
                            r_1r_1r_1^{-1} &= &&r_1\\
                            r_2r_1r_2^{-1} &= r_2^{-1} = &&r_1\\
                            f_1r_1f_1^{-1} &= f_3f_1^{-1} = f_3f_1 = &&r_2\\
                            f_2r_1f_2^{-1} &= f_1f_2^{-1} = f_1f_2 = &&r_2\\
                            f_3r_1f_3^{-1} &= f_2f_3^{-1} = f_2f_3 = &&r_2
                        \end{align*}
                    \item $r_2$:
                        \begin{align*}
                            r_1 r_2 r_1^{-1} &= r_1^{-1} = &&r_2\\
                            r_2 r_2 r_2^{-1} &= r_1r_1 = &&r_2\\
                            f_1 r_2 f_1^{-1} &= f_2 f_1 = &&r_1\\
                            f_2 r_2 f_2^{-1} &= f_3f_2 = &&r_2\\
                            f_3 r_2 f_3^{-1} &= f_1f_3 = &&r_1
                        \end{align*}
                    \item $f_1$:
                        \begin{align*}
                            r_1 f_1 r_1^{-1} &= f_2 r_2 = &&f_3\\
                            r_2 f_1 r_2^{-1} &= f_3 r_1 =&& f_2\\
                            f_1 f_1 f_1^{-1} &= &&f_1\\
                            f_2 f_1 f_2^{-1} &= r_1f_2 =&& f_3\\
                            f_3 f_1 f_3^{-1} &= r_2f_3 = &&f_2
                        \end{align*}
                    \item $f_2$:
                        \begin{align*}
                            r_1 f_2 r_1^{-1} &= r_1 f_3 = &&f_1\\
                            r_2 f_2 r_2^{-1} &= r_2 f_1 = &&f_3\\
                            f_1 f_2 f_1^{-1} &= f_1 r_1 = &&f_3 \\
                            f_2 f_2 f_2^{-1} &= &&f_2\\
                            f_3 f_2 f_3^{-1} &= r_2 f_3 =&& f_2
                        \end{align*}
                    \item $f_3$:
                        \begin{align*}
                            r_1 f_3 r_1^{-1} &= r_1 f_1 = &&f_2\\
                            r_2 f_3 r_2^{-1} &= r_2 f_2 = &&f_1\\
                            f_1 f_3 f_1^{-1} &= f_1 r_2 = &&f_2\\
                            f_2 f_3 f_2^{-1} &= f_2 r_1 = &&f_1\\
                            f_3 f_3 f_3^{-1} &= &&f_3
                        \end{align*}
                \end{enumerate}
            \end{multicols}
            
            So there are three conjugacy classes: $\boxed{\{e\}, \{r_1, r_2\}, \{f_1, f_2, f_3\}}$.

        \color{black}
    
\end{enumerate}
\pagebreak

\section*{6.16b} Give two examples of non-abelian groups with $2^3$ elements.\footnote{In the book there is a proof that all groups of order $p^2$ for prime $p$ are abelian. This shows the proof cannot be extended to groups of order $p^3$.} (Hint. There are only two such groups, up to isomorphism, and both have been used as examples in class before.)

    \color{blue}
        From Example 2.23, the Quaternion Group $Q = \{\pm 1, \pm i, \pm j, \pm k\}$ is a non-commutative group with 8 elements. Additionally, for $n \geq 2$, the dihedral group $D_n$ is a non-commutative group with $2n$ elements. Thus, $D_4$ is a non-abelian group with $8$ elements. $\qed$
    \color{black}

\pagebreak

\section*{6.21} 
Let $p$ be prime, and let $G$ be a group of order $p^n$. Prove that for every $0 \leq r \leq n$, there is a subgroup $H$ of $G$ of order $p^r$. (Hint. Give a proof by induction on $n$. Use Theorem 6.25, which says that $G$ has a non-trivial center $Z(G)$, and apply the induction hypothesis to $G/N$, where $N$ is an appropriately chosen subgroup of $Z(G)$).

    \color{blue}
        We proceed by induction on $n$. For $n = 0$, $G = \{e\}$ and $H = \{e\}$ is a subgroup of order $p^0$. 

        Let $G$ be the smallest group whose order is a multiple of $p$ (say $\#G = p^n$) for which we do not know that there is a subgroup of order $p^4$ for all $0\leq r \leq n$.
        
        By Theorem 6.25, $G$ has a non-trivial center $Z(G)$.
        
        Since $Z(G)$ is a subgroup of $G$, it has order $p^r$ for some $0 < r \leq n$. Then by Abelian Cauchy, there is a subgroup $N \subset Z(G)$ of order $p$. Since $Z(G)$ is the group of elements of $G$ that commute with every other element of $G$, 
        \[gNg^{-1} = \{ghg^{-1} : h\in N\} = \{hgg^{-1}: h \in N\} = N\]
        for all $g \in G$. Thus $N$ is normal in $G$. Further, by Corollary 6.26, 
        \[\#(G/N) = \#G/\#N = p^n/p = p^{n-1}\]
        
        Now we consider the natural homomorphism $\pi: G \to G/N$ defined by $\pi(g) = gN$. This is a $p$-to-$1$ map. If we let $H = \pi^{-1}(G/N) \subset G$ we thus get a subgroup of order $p^{n-1}$ in $G$. 

        By the induction hypothesis, this is a group of order less than $p^n$ so it has subgroups of order $p^r$ for all $0 \leq r \leq n-1$. For the last equality, $G \subseteq G$ trivially so for all $0 \leq r \leq n$, there is a subgroup $H$ of $G$ of order $p^r$. $\qed$
        
    \color{black}


\pagebreak

\section*{6.23}
In Example 6.36 the book shows that there are exactly two groups of order 10. Do a similar calculation to find all groups of order 15.

    \color{blue}
        If $\#G = 15$, Sylow 1 gives us that there are 3-Sylow and 5-Sylow subgroups of $G$. 

        Sylow 3 tells us that the number of distinct $p$-Sylow subgroups ($n$) of $G$ is congruent to $1 \mod p$ and divides $k$.

        We will start with the 3-Sylow subgroups. Then $15 = 3^1 \cdot 5$ so $n \mid 5$ and $n \equiv 1 \mod 3$. Thus, $n = 1$. So there is only one 3-Sylow subgroup of $G$. We will call it $H_3$. 

        Similarly, for the 5-Sylow Subgroups $15 = 5^1 \cdot 3 \implies (n \mid 3) \land (n \equiv 1 \mod 5)$. Thus, $n = 1$. So there is only one 5-Sylow subgroup of $G$. Call it $H_5$.

        By Proposition 2.51, all groups of prime order are cyclic. Thus, the 3-Sylow subgroup is cyclic and the 5-Sylow subgroup is cyclic. Further, by Remark 6.33, $H_3 \cap H_5 = \{e\}$ so 
        \[H_3 = \{e, a, a^2\}, \qquad H_5 = \{e, b, b^2, b^3, b^4\}\]

        When do $a$ and $b$ commute? 
        \[aba^{-1} \in aH_5a^{-1} = H_5 \implies aba^{-1} = b^j \quad 0 \leq j \leq 4\] 
        by normality of $H_5$. What is the value of $j$?

        As in Example 6.36, 
        \[b = a^{-1}b^ja = (a^{-1}ba)^j = (a^{-1}(a^{-1}b^ja)a)^j = a^{-2}b^{j^2}a^2 = a^{-2}(a^{-1}b^ja)^{j^2}a^2 = a^{-3}b^{j^3}a^3 = b^{j^3}\]
        So $b^{j^3 - 1} = e$ and 
        \[j^3 \equiv 1 \mod 5 \implies j = 1\]

        This tells us that $aba^{-1} = b \implies ab = ba$ so $G$ is abelian. Does $ab$ have order 15? 
        \[e = (ab)^k = a^k b^k \implies a^k = b^{-k} \in H_3 \cap H_5 \implies a^k = b^k = e \implies 3 \mid k \text{ and } 5 \mid k \implies 15 \mid k\]
        So $G$ is a cyclic group of order 15 and all groups of order 15 are isomorphic to $\mathcal{C}_{15}$. $\qed$

    \color{black}


\pagebreak

\section*{6.25} 
Let $G$ be a finite group of order $\#G = pq$, where $p$ and $q$ are primes satisfying $p > q$. Assume further that $p \not \equiv 1 (\mod q)$
\begin{enumerate}
    \item Prove that $G$ is an abelian group. (Hint. Example 6.37 provides a starting point for the proof.)

        \color{blue}
            Let $n_p$ be the number of $p$-Sylow subgroups and $n_q$ the number of $q$-Sylow subgroups. 
            
            By Sylow 3, $n_q \equiv 1 \mod q$ and $n_q \mid p$. Since $p$ is prime, $n_q \in \{1, p\}$ but $p \not \equiv 1 \mod q$ so $n_q = 1$. Thus, there is only one $q$-Sylow subgroup of $G$ and it is normal. Further, since $p > q$, $q$ is the largest power of itself which divides $\#G$ so the order of the $q$-Sylow subgroup is $q$. 

            Now look at the $p$-subgroups. Let $H$ be one of them. By Example 6.37, $H$ is normal. Let $H'$ be another $p$-Sylow subgroup. Then by Sylow 2, 
            \[H = aH'a^{-1} \quad a\in G\]
            but since $H$ is normal, 
            \[H= aHa^{-1}\]
            so 
            \[aHa^{-1} = aH'a^{-1} \implies H = H'\]
            so there is only one $p$-Sylow subgroup.           

            Call the $p$-Sylow subgroup $A$ and the $q$-Sylow subgroup $B$. Then consider the set of products $AB = \{ab : a \in A, b \in B\}$. Since $p$ and $q$ are distinct, $A \cap B = \{e\}$. By Lemma 6.39, 
            \[\#(AB) = \frac{\#A \cdot \#B}{\#(A \cap B)} = \frac{p \cdot q}{1} = pq = \#G \implies G = AB\]
            and further, we have a well-defined bijection $\frac{A \times B}{A \cap B} \to AB$. 

            Thus, 
            \[G = \{a^i b^j : 0 \leq i \leq p - 1, \; 0 \leq j \leq q - 1\}\]
           
            To show that two elements of $G$ commute, then, it is sufficient to show that elements of $A$ and $B$ commute. Consider the product $hgh^{-1}g^{-1}$ with $h \in A$ and $g \in B$.

            Since $B$ is normal, $hgh^{-1} \in A$ so $(hgh^{-1})g^{-1} \in A$. But since $A$ is also normal, $gh^{-1}g^{-1} \in A \implies h(gh^{-1}g^{-1}) \in A$. Thus, $hgh^{-1}g^{-1} \in A \cap B$. But $A \cap B = \{e\}$ so $hgh^{-1}g^{-1} = e \implies hg = gh$. Thus, $G$ is abelian. $\qed$

        \color{black}

    \item Prove that $G$ is a cyclic group.

        \color{blue}
            Let $A = \brak{a}$ be the $p$-Sylow subgroup and $B = \brak{b}$ be the $q$-Sylow subgroup as above. By Example 6.36, to show that $G$ is cyclic, we just need to show that it is abelian and that $ab$ has order $pq$. 

            We have already shown that $G$ is abelian so all the remains is to show that $ab$ has order $pq$: 
            \begin{align*}
                e &= (ab)^k = a^kb^k\\
                &\implies a^k = b^{-k} \in A \cap B = \{e\} \\
                &\implies a^k = b^k = e \\
                &\implies p \mid k \text{ and } q \mid k \\
                &\implies pq \mid k
            \end{align*}

            Thus the smallest number for which $(ab)^k = e$ is $pq$ so the order of $ab$ is $pq$ and $G$ is cyclic. $\qed$ 
        \color{black}

\end{enumerate}

\pagebreak

\section*{6.28} Let $G$ be a group of order $\#G = 75$.
\begin{enumerate}
    \item Prove that $G$ has a subgroup $H$ with all three of the following properties:
    \begin{itemize}
        \item $H$ has order $\#H = 25$.

            \color{blue}
                $25 = 5^2$ and $5 \mid 75$ but $5^3 \not \mid 75$ so by Sylow 1, $G$ has at least one 5-Sylow subgroup $H$ of order 25. $\qed$
            \color{black}

        \item $H$ is a normal subgroup of $G$.
                
            \color{blue}
                By Sylow 3, the number of distinct 5-Sylow subgroups of $G$ is congruent to $1 \mod 5$ and divides $3$. The divisors of $3$ are $\{1, 3\}$. Of these, only $1 \equiv 1 \mod 5$ so $H$ is the only 5-Sylow subgroup of $G$.

                Let $S$ be the set of Sylow p-subgroups. By Sylow 2, $G$ acts on $S$ by conjugation: $g\cdot H = gHg^{-1}$. In class we showed that 
                \[\frac{\#G}{\#S_H} = \#S\]
                
                Since $H$ is the only 5-Sylow subgroup of $G$, $\#S = 1$. Thus, 
                \[S_H = \{a \in G \; | \; aHa^{-1} = H\} = G\] 
                so $H$ is normal. $\qed$
            \color{black}
            
        \item $H$ is abelian.
        
            \color{blue}
                $H$ is a group with $25 = 5^2$ elements so by Corollary 6.26, $H$ is abelian. $\qed$
            \color{black}
    \end{itemize}
    
    \item Suppose that the subgroup $H$ in (a.) is cyclic of order 25. Prove that $G$ is abelian.
    
        \color{blue}
            We know from above that $H$ is a unique 5-Sylow subgroup of $G$. 
            
            Now consider the $3$-Sylow subgroups: 
            \[n \mid 25, \quad n \equiv 1 \mod 3 \implies n = 1\]
            so there is a unique 3-Sylow subgroup of order $3$ in $G$. 

            Let $K$ be the 3-Sylow subgroup. Then $H \cap K = \{e\}$ and by Lemma 6.39,
            \[\#(HK) = \frac{\#H \cdot \#K}{\#(H \cap K)} = \#H \cdot \#K = 25 \cdot 3 = 75\]

            Though $HK$ is not generally a subgroup of $G$, it is a subset since $H$ and $K$ are cyclic subgroups of $G$ so all their elements are elements of $G$ and thus by closure, the set of products is in $G$. This further tells us that since $HK \subseteq G$ and $\#HK = \#G$ (and $G$ is finite), $HK = G$, i.e., 
            \[G = \{h^i k^j : h^i \in H, k^j \in K\}\]

            To show that $G$ is abelian, it is sufficient to show that $H$ and $K$ commute. Consider the product $hkh^{-1}k^{-1}$ with $h \in H$ and $k \in K$.

            Since both $H$ and $K$ are normal, 
            \[\underbrace{(hkh^{-1})}_{\in K}k^{-1} = h\underbrace{(kh^{-1}k^{-1})}_{\in H} \implies hkh^{-1}k^{-1} \in H \cap K = \{e\}\] 
            so 
            \[hkh^{-1}k^{-1} = e \implies hk = kh\]
            so $G$ is abelian. $\qed$
        \color{black}
    
\end{enumerate}


\end{document}