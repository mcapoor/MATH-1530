\documentclass[12pt]{article} 
\usepackage[utf8]{inputenc}
\usepackage{geometry}
\geometry{letterpaper}
\usepackage{graphicx} 
\usepackage{parskip}
\usepackage{booktabs}
\usepackage{array} 
\usepackage{paralist} 
\usepackage{verbatim}
\usepackage{subfig}
\usepackage{fancyhdr}
\usepackage{sectsty}
\usepackage[shortlabels]{enumitem}

\pagestyle{fancy}
\renewcommand{\headrulewidth}{0pt} 
\lhead{}\chead{}\rhead{}
\lfoot{}\cfoot{\thepage}\rfoot{}


%%% ToC (table of contents) APPEARANCE
\usepackage[nottoc,notlof,notlot]{tocbibind} 
\usepackage[titles,subfigure]{tocloft}
\renewcommand{\cftsecfont}{\rmfamily\mdseries\upshape}
\renewcommand{\cftsecpagefont}{\rmfamily\mdseries\upshape} %

\usepackage{amsmath}
\usepackage{amssymb}
\usepackage{mathtools}
\usepackage{empheq}
\usepackage{xcolor}

\usepackage{tikz}
\usepackage{pgfplots}
\pgfplotsset{compat=1.18}

\newcommand{\ans}[1]{\boxed{\text{#1}}}
\newcommand{\vecs}[1]{\langle #1\rangle}
\renewcommand{\hat}[1]{\widehat{#1}}
\newcommand{\F}{\mathbb{F}}
\renewcommand{\P}{\mathbb{P}}
\newcommand{\R}{\mathbb{R}}
\newcommand{\C}{\mathbb{C}}
\newcommand{\E}{\mathbb{E}}
\newcommand{\Z}{\mathbb{Z}}
\newcommand{\ind}{\mathbbm{1}}
\newcommand{\qed}{\quad \blacksquare}
\newcommand{\brak}[1]{\left\langle #1 \right\rangle}
\newcommand{\bra}[1]{\left\langle #1 \right\vert}
\newcommand{\ket}[1]{\left\vert #1 \right\rangle}
\newcommand{\abs}[1]{\left\vert #1 \right\vert}
\newcommand{\mfX}{\mathfrak{X}}

\title{Math 1530: Homework 9}
\author{Milan Capoor}
\date{5 December 2023}

\begin{document}
\maketitle
\section*{7.1}
Let $R$ be an integral domain, and let $a, b \in R$. Prove that the following are equivalent:
\begin{enumerate}[(a)]
    \item $a \mid b$ and $b \mid a$.
    \item There is a unit $u \in R^*$ satisfying $a = bu$.
    \item There is an equality of principal ideals $aR = bR$
\end{enumerate}

    \color{blue}
        By Definition 7.3, if $a \mid b$ then $bR \subseteq aR$. Similarly, if $b \mid a$ then $aR \subseteq bR$. Thus, $aR = bR$. So $(a) \implies (c)$. 

        Now suppose $aR = bR$. Then $a \in bR \implies a = bu \quad (u \in R)$ and $b \in aR \implies b = ac \quad (c \in R)$. Thus, 
        \[a = bu = (ac)u \implies cu = 1\]
        Which means that $u$ is a unit satisfying $a = bu$ and $(c) \implies (b)$ 
        
        Finally, suppose we have a unit $u \in R^*$ with $a = bu$. Clearly, $b \mid a$. But by definition, there exists some $v \in R^*$ such that $uv = 1$. And so we can write 
        \[a = bu \implies a\cdot 1 = bu \cdot 1 \implies auv = bu \implies av = b \implies a \mid b\]
        because $R$ is an integral domain. Thus, $(b) \implies (a)$ and we have a loop of equivalences 
        \[(a) \to (c) \to (b) \to (a)\]
        so we are done. $\qed$
    \color{black}

\pagebreak

\section*{7.4}
Let $R$ be a Euclidean domain with size function $\sigma$, and let $a \in R$ be a non-zero element of $R$. Prove that
\[a\in R^* \iff \sigma(a) = \sigma(1)\]

    \color{blue}
        If $a \in R^*$, then $a$ is a unit. Thus, there exists some nonzero $b \in R$ such that $ab = 1$. By Proposition 7.15
        \[\sigma(b) = \sigma(ab) = \sigma(1)\]
        
        Further, by the properties of a size function, 
        \[\sigma(a) \leq \sigma(ab) = \sigma(1)\] 

        Now consider the ideal $I = (1) = R$. By Corollary 7.11, $\sigma(1) \leq \sigma(r)$ for every non-zero $r \in I = R$. Thus, 
        \[\sigma(a) \leq \sigma(1) \leq \sigma(a) \implies \sigma(a) = \sigma(1) \quad \square\]
        
        Now for the other direction, suppose $\sigma(a) = \sigma(1)$. Then, since $a \neq 0$, we can write 
        \[1 = aq + r\]
        with $\sigma(r) < \sigma(a)$. 

        But 
        \[\sigma(a) = 1 \implies \sigma(r) = 0 \implies 1 = aq\] 
        so $a$ is a unit. $\qed$
    \color{black}
\pagebreak

\section*{7.6}
For a complex number $z = x + yi \in \C$, let $\sigma(z)$ be the square of its norm,
\[\sigma(z) = \sigma(x + yi) = x^2 + y^2\]
\begin{enumerate}[(a)]
    \item Prove that $\sigma(z) = 0$ if and only if $z = 0$.

        \color{blue}
            If $z = 0$, then $\sigma(z) = 0^2 + 0^2 = 0$. 
            
            Now suppose $\sigma(z) = 0$. Then $x^2 + y^2 = 0$. But $x^2, y^2 \geq 0$ so $x^2 + y^2 \geq 0$. So the only way for $\sigma(z)$ to be $0$ is if $x^2 = y^2 = 0$. Thus,  $z = 0$. $\qed$
        \color{black}

    \item Prove that $\sigma(z_1z_2) = \sigma(z_1)\sigma(z_2)$.
    
        \color{blue}
            Let $z_1 = a + bi$ and $z_2 = c + di$. Then 
            \begin{align*}
                \sigma(z_1z_2) &= \sigma((a + bi)(c + di))\\ 
                    &= \sigma((ac - bd) + (ad + bc)i)\\
                    &= (ac - bd)^2 + (ad + bc)^2\\
                    &= a^2c^2 - 2abcd + b^2d^2 + a^2d^2 + 2adcb + b^2c^2\\
                    &= a^2c^2 + b^2d^2 + a^2d^2 + b^2c^2\\
                    &= (a^2 + b^2)(c^2 + d^2)\\
                    &= \sigma(z_1)\sigma(z_2) \qed
            \end{align*}
        \color{black}
    
    \item Is $\sigma(z_1 + z_2) = \sigma(z_1) + \sigma(z_2)$? If not, give a counterexample. Can you find an inequality relating these three quantities? 
    
        \color{blue}
            Again let $z_1 = a + bi$ and $z_2 = c + di$. Then
            \begin{align*}
                \sigma(z_1 + z_2) &= \sigma((a + bi) + (c + di))\\
                    &= \sigma((a + c) + (b + d)i)\\
                    &= (a + c)^2 + (b + d)^2\\
                    &= a^2 + 2ac + c^2 + b^2 + 2bd + d^2\\
                    &= (a^2 + b^2) + (c^2 + d^2) + 2(ac + bd)\\
                    &= \sigma(z_1) + \sigma(z_2) + 2(ac + bd)
            \end{align*}
            
            So $\sigma(z_1 + z_2) = \sigma(z_1) + \sigma(z_2) + 2(ac + bd)$. Thus, $\sigma(z_1 + z_2) = \sigma(z_1) + \sigma(z_2)$ if and only if $ac + bd = 0$.

            For example, let $z_1 = 2 + i$ and $z_2 = 1 + i$. Then
            \begin{align*}
                \sigma(z_1 + z_2) &= \sigma(3 + 2i) = 9 + 4 = 13\\
                \sigma(z_1) + \sigma(z_2) &= \sigma(2 + i) + \sigma(1 + i) = 5 + 2 = 7
            \end{align*}
            so $\sigma(z_1) + \sigma(z_2) \leq \sigma(z_1 + z_2)$.

            However, for $z_1 = 1 + i$ and $z_2 = -1 + i$, we have
            \[\sigma(z_1 + z_2) = 4 = \sigma(z_1) + \sigma(z_2) = 4\]
            so $\sigma(z_1 + z_2) = \sigma(z_1) + \sigma(z_2)$.

            And finally, with $z_1 = 2 + i$ and $z_2 = -2 - i$, we have
            \begin{align*}
                \sigma(z_1 + z_2) = 0\\ 
                \sigma(z_1) + \sigma(z_2) = 5 + 5 = 10
            \end{align*}
            so $\sigma(z_1 + z_2) \leq \sigma(z_1) + \sigma(z_2)$.
        \color{black}
\end{enumerate}
\pagebreak

\section*{7.8}
For each of the following rings $R$ and elements $\alpha$, determine whether $\alpha$ is irreducible in $R$. Justify your answer by either factoring $\alpha$, or proving that it is irreducible. (Hint. For (a) and (b), it might be helpful to use the size function for $\Z[i]$.)
\begin{enumerate}[(a)]
    \item $R = \Z[i]$, $\alpha = 2 + 3i$
    
        \color{blue}
            First note that $\sigma(2 + 3i) = 2^2 + 3^2 = 13$. Let $\alpha = \zeta_1 \zeta_2$. By Exercise 7.6, $\sigma(\zeta_1 \zeta_1) = \sigma(\zeta_1)\sigma(\zeta_2)$ so 
            \[\sigma(\alpha) = \sigma(\zeta_1)\sigma(\zeta_2) = 13\]
            But this means that $\sigma(\zeta_1) = 1$ and $\sigma(\zeta_2) = 13$ or vice versa. However, from exercise 7.4, we know that $\sigma(\zeta) = 1$ if and only if $\zeta$ is a unit. So $\alpha$ has a unit factor and is thus irreducible. $\qed$
        \color{black}

    \item $R = \Z[i]$, $\alpha = 4 + 3i$

        \color{blue}
           \[\alpha = (1 + 2i)(2 - i)\]
        \color{black}

    \item $R = \F_2[x]$, $\alpha = x^5 + x + 1$
    
        \color{blue}
            \[\alpha = (x^2 + x + 1)(x^3 - x^2 + 1)\]
        \color{black}

    \item $R = \F_2[x]$, $\alpha = x^5 + x^2 + 1$
    
        \color{blue}
            From example 7.13, $R$ is a Euclidean Domain so it is a PID. Consider the principal ideal $I = \alpha R$. 

            Consider two elements $a, b \in R$. Then $(a + I)(b + I) = ab + I$. Suppose $ab = 0$. But since 
            \[ab = \left(\sum_{i=1}^n a_ix^i\right)\left(\sum_{j=1}^m b_ix^j\right)\]
            and $a_i, b_i \in \F_2$, every coefficient will be $0$ or $1$. Chiefly, this means that there will be no negative coefficients so the only way for the product to be zero is if one of the factors is $0$. Hence, $\F_2[x]/\alpha F_2[x]$ is an integral domain. Then by  Theorem 3.43, $\alpha R$ is prime so by Theorem 7.16, $\alpha$ is irreducible. $\qed$
        \color{black}

\end{enumerate}
\pagebreak

\section*{7.12}
Let $R = \Z[\sqrt 3]$ be the ring that we studied in Example 7.21.
\begin{enumerate}[(a)]
    \item Prove that $\pm 1$ are the only units in $R$

        \color{blue}
            Assume $a + b\sqrt{-3}$ is a unit in $R$. Then there exists some $c + d\sqrt{-3}$ such that
            \[(a + b\sqrt{-3})(c + d\sqrt{-3}) = 1 \implies \begin{cases}
                ac - 3bd = 1\\
                ad + bc = 0
            \end{cases}\]
            We can solve this system of equations via 
            \[\begin{pmatrix}
                a & -3b\\
                b & a
            \end{pmatrix} \begin{pmatrix}
                c\\
                d
            \end{pmatrix} = \begin{pmatrix}
                1\\
                0  
            \end{pmatrix} \implies \begin{pmatrix}
                c\\
                d
            \end{pmatrix} = \frac{1}{a^2 + 3b^2}\begin{pmatrix}
                a\\
                -b 
            \end{pmatrix}\]
            so 
            \[(a + b\sqrt{-3})(\frac{a}{a^2 + 3b^2} - \frac{b}{a^2 + 3b^2}\sqrt{-3}) = 1\]
            
            $\abs{a} = \abs{a^2 + 3b^2}$ only in the cases $a, b = 0$ and $a = \pm 1, b = 0$. In the first case we have $0 = 1$ which is false and in the second, we get $1 = 1$ which is true. In any other case, the quantity $\frac{a}{a^2 + 3b^2} \not \in \Z$ so the only units in $R$ are $\pm 1$. $\qed$ 
        \color{black}

    \item Prove that $1 + \sqrt{-3}$ and $1 - \sqrt{-3}$ are irreducible elements of $R$. (Hint. After you prove that $1 + \sqrt{-3}$ is irreducible, you can use Exercise 3.5(b) to prove that $1 - \sqrt{-3}$ is also irreducible.)
            
        \color{blue}
            By part (a), they are not units. 

            Now we want to show that $1 + \sqrt{-3}$ has a unit factor. Consider 
            \[1 + \sqrt{-3} = (a + b\sqrt{-3})(c + d\sqrt{-3})\]

            Expanding, 
            \[1 + \sqrt{-3} = (ac - 3bd) + (ad + bc)\sqrt{-3} \implies \begin{cases}
                ac - 3bd = 1\\
                ad + bc = 1
            \end{cases}\]
            solving for $c$ and $d$ gives
            \[\begin{pmatrix}
                a & -3b\\
                b & a
            \end{pmatrix} \begin{pmatrix}
                c\\
                d
            \end{pmatrix} = \begin{pmatrix}
                1\\
                1
            \end{pmatrix} \implies \begin{pmatrix}
                c\\
                d
            \end{pmatrix} = \begin{pmatrix}
                \frac{a + 3b}{a^2 + 3b^2}\\
                \frac{a - b}{a^2 + 3b^2}
            \end{pmatrix}\]

            We want both of these to be solvable in the integers so 
            \[a - b = \lambda(a^2 + 3b^2) \quad \lambda \in \Z \implies \abs{a^2 + 3b^2} \leq \abs{a - b}\]

            This is clearly true at $a, b = 0$ and $a = \pm 1, b = 0$ but $0 \neq 1 + \sqrt{-3}$ and
            \[(1 + 0\sqrt{-3})(\frac{1 + 3(0)}{1 + 0}) = (1)(1) \neq 1 + \sqrt{-3}\] 
            In any case where $b > a$, the LHS is greater than the RHS. So we must have $b \leq a$. But at $a = 2, b = 1$, the LHS is already too large. Thus, there are no solutions in the integers and $1 + \sqrt{-3}$ is irreducible.

            From exercise 3.5, we have a homomorphism $\phi(a + b\sqrt{-3}) = a - b\sqrt{-3}$. Suppose $1 - \sqrt{-3}$ is reducible. Then there exists some $a + b\sqrt{-3}$ such that
            \[1 - \sqrt{-3} = (a + b\sqrt{-3})(c + d\sqrt{-3})\]
            but since $\phi$ is a homomorphism from $\Z[\sqrt{-3}] \to \Z[\sqrt{-3}]$,
            \[1 - \sqrt{-3} = (a + b\sqrt{-3})(c + d\sqrt{-3}) \implies \phi^{-1}(1 - \sqrt{-3}) = 1 + \sqrt{-3} = \phi^{-1}(a + b\sqrt{-3})\phi^{-1}(c + d \sqrt{-3})\]
            But we just showed that $1 + \sqrt{-3}$ is irreducible so this is a contradiction. Thus, $1 - \sqrt{-3}$ is irreducible. $\qed$

        \color{black}
    
    \item Prove that $2$ is an irreducible element of $R$. (We already proved this in the text, but try doing it yourself without looking back.)
    
        \color{blue}
            By part (a), $2$ is not a unit.

            Does it factor into units? Consider, 
            \[2 = (a + b\sqrt{-3})(c + d\sqrt{-3}) = (a - b\sqrt{-3})(c - d\sqrt{-3})\]
            Multiplying, 
            \[4 =  (a + b\sqrt{-3})(c + d\sqrt{-3})(a - b\sqrt{-3})(c - d\sqrt{-3}) = (a^2 + 3b^2)(c^2  + 3d^2)\]
            with $a, b, c, d \in \Z$. 

            We know the only integer factorizations are $4 = 2 \cdot 2 = 4 \cdot 1$. 

            Checking the first case,
            \[\begin{cases}
                a^2 + 3b^2 = 2\\
                c^2 + 3d^2 = 2
            \end{cases}\]
            we find no solutions in the integers because if $b > 1$, then the LHS is greater than two and $a = \sqrt{2} \not \in \Z$. Similar argument gives no solutions to the second case.

            Thus, we try 
            \[\begin{cases}
                a^2 + 3b^2 = 4\\
                c^2 + 3d^2 = 1
            \end{cases}\]
            (or vice versa). The second line gives $c = \pm 1$ and $d = 0$ (or $a = \pm 1$ and $b = 0$). But then one of the factors of two is a unit, so it is irreducible. $\qed$

        \color{black}

    \item Use (a), (b), (c) and the formula 
    \[4 = 2 \cdot 2 = (1 + \sqrt{-3})(1 - \sqrt{-3})\]
    to deduce that $R$ is not a UFD.

        \color{blue}
            From (a), (b), and (c), we have that $2$, $1 + \sqrt{-3}$, $1 - \sqrt{-3}$ are irreducible elements of $R$. The formula shows that $4$ has two distinct factorizations into irreducibles. Thus, $R$ is not a UFD. $\qed$

        \color{black}
\end{enumerate}

\end{document}