\documentclass[12pt]{article} 
\usepackage[utf8]{inputenc}
\usepackage{geometry}
\geometry{letterpaper}
\usepackage{graphicx} 
\usepackage{parskip}
\usepackage{booktabs}
\usepackage{array} 
\usepackage{paralist} 
\usepackage{verbatim}
\usepackage{subfig}
\usepackage{fancyhdr}
\usepackage{sectsty}
\usepackage{enumitem}

\pagestyle{fancy}
\renewcommand{\headrulewidth}{0pt} 
\lhead{}\chead{}\rhead{}
\lfoot{}\cfoot{\thepage}\rfoot{}


%%% ToC (table of contents) APPEARANCE
\usepackage[nottoc,notlof,notlot]{tocbibind} 
\usepackage[titles,subfigure]{tocloft}
\renewcommand{\cftsecfont}{\rmfamily\mdseries\upshape}
\renewcommand{\cftsecpagefont}{\rmfamily\mdseries\upshape} %

\usepackage{amsmath}
\usepackage{amssymb}
\usepackage{mathtools}
\usepackage{empheq}
\usepackage{xcolor}

\usepackage{tikz}
\usepackage{pgfplots}
\pgfplotsset{compat=1.18}

\newcommand{\ans}[1]{\boxed{\text{#1}}}
\newcommand{\vecs}[1]{\langle #1\rangle}
\renewcommand{\hat}[1]{\widehat{#1}}
\newcommand{\F}[1]{\mathcal{F}(#1)}
\renewcommand{\P}{\mathbb{P}}
\newcommand{\R}{\mathbb{R}}
\newcommand{\E}{\mathbb{E}}
\newcommand{\Z}{\mathbb{Z}}
\newcommand{\ind}{\mathbbm{1}}
\newcommand{\qed}{\quad \blacksquare}
\newcommand{\brak}[1]{\langle #1 \rangle}
\newcommand{\bra}[1]{\langle #1 |}
\newcommand{\ket}[1]{| #1 \rangle}

\title{Midterm 1 Review}
\author{Milan Capoor}
\date{10 Oct 2023}

\begin{document}
\maketitle
\section*{List of Topics}
\textbf{Groups:}
\begin{enumerate}
    \item Basic Definition 
        \begin{itemize}
            \item Symmetric Group
            \item Permutation groups 
            \item Dihedral groups 
            \item Complex numbers
            \item Cyclic groups
        \end{itemize}
    \item Subgroups
        \begin{itemize}
            \item Cyclic Subgroups
            \item Subgroups generated by a set
        \end{itemize}
    \item Homomorphisms 
    \item The Kernel
    \item Cosets 
    \item Lagrange's Theorem 
    \item Euler's Theorem for $(\Z/n)^*$ 
    \item RSA
\end{enumerate}

\textbf{Rings:}
\begin{enumerate}
    \item Basic definition 
        \begin{itemize}
            \item Matrices 
            \item Polynomials 
            \item $\Z/n$
            \item Products 
            \item 
        \end{itemize}
    \item Ideals
    \item Homomorphisms 
    \item Group of units
    \item Types of rings 
        \begin{itemize}
            \item Fields
            \item Integral domains
        \end{itemize}
    \item Cosets 
    \item Quotient construction
    \item Isomorphism theorem 
\end{enumerate}

\section*{Groups}
\subsection*{Dihedral Group}
\textbf{Definition:} the group of symmetries of regular polygons 

\emph{Example:}
\begin{itemize}
    \item $D_3 = \{e, r, r^2, f, f^2, f^3\}$
\end{itemize}

\textbf{Order:} the order of $D_n$ is $2n$ 

\textbf{Subgroup:} the subgroup of rotations of $D_n$ are isomorphic to the cyclic group $C_n$ and (equivalently) to $\Z/n$

\subsection*{Groups of prime order}
Have no subgroups other than $\{e\}$ and $G$

\subsection*{Groups generated by multiple elements}
$\brak{g_1, \dots,\, g_k}$ = $\{$all ``words'' in $g_1, \dots,\, g_k$ and their inverses$\}$ = the intersection of all subgroups that contain $g_1, \dots,\, g_k$

\subsection*{Cosets}
$H \subset G$. Look at all $aH$ 
\[H \to aH, \quad h \mapsto ah\]

These provide a partition of the group by cosets of the same size. 

We also know that all cosets are identical or disjoint.

\subsection*{Euler's Theorem}
\[\phi(n) = \#(\Z/n\Z)^*\]
where $(\Z/n\Z)^*$ is the group of units (all the elements with inverses) of $\Z/n\Z$ which is the same as all the elements that are relatively prime to $n$. 

\emph{Example:} $(\Z/8)^* = \{1, 3, 5, 7\} \longrightarrow \phi(8) = 4$

By Lagrange's theorem, if $a \in (\Z/n\Z)^*$ then $o(a) \bigg\vert \phi(n)$ or $a^{\phi(n)/d} = 1$ in $(\Z/n\Z)^*$. So 
\[a^{\phi(n)} \equiv 1 \mod n \]
(if $\gcd(a, n) = 1$)

\section*{Rings}
\subsection*{Ideal}
For a commutative ring $R$, $I \subset R$ is an ideal when $(I, +)$ is an abelian subgroup and it has the \textbf{absorber property} 
\[ar \in I, \qquad \forall a \in I, r\in R\]

\textbf{Cosets:} $\{a + I, a\in R\}$ 
\emph{The only subgroup that is a coset is $0 + I$}

\subsection*{Homomorphism}
A ring homomorphism is also a group homomorphism (because a ring is an ``enhancement'' of an abelian group)

The kernel of a group is a subgroup. The kernel of a ring is a subring, but more strongly, an ideal. 

\subsection*{Group of units}
\textbf{Unit:} an element in a ring with an inverse 

\emph{Proof of group:}
\begin{gather}
    aa^{-1} = 1\\ 
    bb^{-1} = 1\\ 
    ab \cdot a^{-1}b^{-1} = 1
\end{gather}

\textbf{Field:} if every non-zero element is a unit 

\subsection*{Quotient}
Given an ideal $I$ and ring $R$, $R/I$ is the set of cosets $\{a + I, \; a\in R\}$ with
\begin{align*}
    (a + I) + (b + I) &= (a + b) + I\\ 
    (a + I)(b + I) &= (ab) + I
\end{align*}
assuming that the above are well-defined.

The first formula comes from the fact $I$ is a group. For the second, observe 
\begin{gather}
    a' = a + i_1 \quad b' = b + i_2\\ 
    a'b' = (a + i_1)(b + i_2) = ab + ai_2 + bi_2 + i_1i_2
\end{gather}
And by the absorber property, all the products with $i$ are in $I$ so 
\[a'b' = ab + I\]
so the cosets are the same. 

\subsection*{Isomorphism theorem}
We have a homomorphism $\phi: R_1 \to R_2$ which may or may not be onto. We also have 
\[I = \ker(\phi) = \phi^{-1}(0) = \{r \in R_1: \phi(r) = 0\}\]

\begin{center}
    \begin{tikzpicture}
        \node (R1) at (0,5) {$R_1$};
        \node (R2) at (0,0) {$R_1/I$};
        \node (R1/I) at (5,5) {$R_2$};

        \draw[->] (R1) -- (R2) node[midway, left] {$\phi$};
        \draw[->] (R1/I) -- (R2) node[midway, right] {$\alpha$};
        \draw[->] (R1) -- (R1/I) node[midway, above] {$\pi$};
    \end{tikzpicture}
\end{center}

By definition, $\alpha(a + I) = \phi(a)$

If $\alpha(a + I) = 0$, then $\phi(a) = 0$ so $a \in I \implies a + I = 0 + I$. This shows that the kernel is trivial so the map is injective. 

\textbf{Theorem:} the map $\alpha: R_1/I \to R_2$ is a ring isomorphism

Generally, given $R_1/I$ you guess an $R_2$ and look for a homomorphism from $R_1$ to $R_2$ which is onto and then compute the kernel. 

\textbf{Example 1:} Show that $\R[x]/(x^2 + 1)R[x]$ is isomorphic to the complex numbers. 

We find a typical member of $R_1 = \R[x]$:
\[\sum a_k x^k\]

And a typical member of $\mathbb C = \R[i]$
\[\sum a_k i^k\]

So 
\[\phi: R_1 \to R_2 \qquad \sum a_k x^k \mapsto a_k i^k\]

This is onto because to get any value $a + bi$ we just need to see $a + bx$. 

Then to calculate the kernel, notice 
\[p(x) = a + bx + (x^2 + 1)q(x)\]
which is just the analog of the division algorithm. 

We want this to be in the kernel so 
\[0 = p(i) = a + bi + 0q(x)\]
which implies $a, b = 0$ so 
\[p(x) = (x^2 + 1)q(x) = I\]

\textbf{Example 2:} $R_1 = \mathbb Q[x]$ and $I = \mathbb Q[x](x^2 - 5)$. What is $R_1/I$?

We notice that $x^2 - 5 = 0 \implies x = \sqrt 5$ so we guess $\mathbb{Q}[\sqrt 5]$. Now we want an onto ring homomorphism from $\mathbb Q[x] \to \mathbb{Q}[\sqrt 5]$. So 
\[\sum a_k x^k \to \sum a_k (\sqrt 5)^k\]
This is surjective because 
\[a + bx \mapsto a + b\sqrt 5\]

Then 
\[p(x) = a + bx + (x^2 -5)q(x)\]
so $p(x) = I$ when $a$ and $b$ are $0$. 
\end{document}
