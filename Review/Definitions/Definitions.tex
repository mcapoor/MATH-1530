\documentclass[12pt]{article} 
\usepackage[utf8]{inputenc}
\usepackage{geometry}
\geometry{letterpaper}
\usepackage{graphicx} 
\usepackage{parskip}
\usepackage{booktabs}
\usepackage{array} 
\usepackage{paralist} 
\usepackage{verbatim}
\usepackage{subfig}
\usepackage{fancyhdr}
\usepackage{sectsty}
\usepackage{enumitem}

\pagestyle{fancy}
\renewcommand{\headrulewidth}{0pt} 
\lhead{}\chead{}\rhead{}
\lfoot{}\cfoot{\thepage}\rfoot{}


%%% ToC (table of contents) APPEARANCE
\usepackage[nottoc,notlof,notlot]{tocbibind} 
\usepackage[titles,subfigure]{tocloft}
\renewcommand{\cftsecfont}{\rmfamily\mdseries\upshape}
\renewcommand{\cftsecpagefont}{\rmfamily\mdseries\upshape} %

\usepackage{amsmath}
\usepackage{amssymb}
\usepackage{mathtools}
\usepackage{empheq}
\usepackage{xcolor}

\usepackage{tikz}
\usepackage{pgfplots}
\pgfplotsset{compat=1.18}

\newcommand{\ans}[1]{\boxed{\text{#1}}}
\newcommand{\vecs}[1]{\langle #1\rangle}
\renewcommand{\hat}[1]{\widehat{#1}}
\newcommand{\F}[1]{\mathcal{F}(#1)}
\renewcommand{\P}{\mathbb{P}}
\newcommand{\R}{\mathbb{R}}
\newcommand{\E}{\mathbb{E}}
\newcommand{\Z}{\mathbb{Z}}
\newcommand{\ind}{\mathbbm{1}}
\newcommand{\qed}{\quad \blacksquare}
\newcommand{\brak}[1]{\langle #1 \rangle}
\newcommand{\bra}[1]{\langle #1 |}
\newcommand{\ket}[1]{| #1 \rangle}

\begin{document}
\section*{Groups}
\textbf{Group:} a set with a composition law satisfying
\begin{align*}
    g_1 g_2 \in G \quad &\forall g_1, g_2 \in G \qquad &(\text{closure})\\
    eg = ge = g \quad &\forall g\in G \qquad &(\text{identity})\\
    gh = hg = e \quad &\exists h \forall g \in G \qquad &(\text{inverse})\\ 
    g_1 (g_2g_3) = (g_1g_2)g_3 \quad &\forall g_1, g_2, g_2 \in G \qquad &(\text{associativity})
\end{align*}

\emph{Properties:}
\begin{itemize}
    \item G has exactly one identity
    \item Each element has exactly one inverse 
    \item $(gh)^{-1} = h^{-1} g^{-1}$
    \item $(g^{-1})^{-1} = g$
    \item If $g_1g_2 = g_2g_1 \quad \forall g_1, g_2 \in G$ then $G$ is \emph{abelian}
    \item \emph{Cancellation:} If $gh = gk$ then $h = k$
\end{itemize}

\textbf{Order:} 
\begin{enumerate}
    \item the cardinality of the set of elements of an infinite group 
    \item the number of elements in a finite group 
    \item the smallest integer $n \geq 1$ such that $g^n = e$ with $g \in G$
    \item the order of a group is also the order of its generator
\end{enumerate}

\textbf{Common groups:}
\begin{itemize}
    \item Cyclic group: $\mathcal{C}_n = \{e, g, g^2, \dots,\, g^{n-1}\} \cong \brak{g}$ using $g^{i} \cdot g^j = g^{i + j\mod n}$
    \item Symmetric group: $\mathcal{S}_n$ the set of all permutations of $\{1, 2, \dots,\, n\}$
    \item General Linear Group $GL_2(\R)$: the group of $2\times 2$ matrices with $\det A \neq 0$
    \item Dihedral groups $\mathcal{D}_n$: the group of rotations and reflections of an $n$-gon 
    \item Quaternion group $\mathcal Q = \{\pm 1, \pm i, \pm j, \pm k\}$
\end{itemize}

\textbf{Group homomorphism:}
\[\phi: G \to G' \;\text{ such that }\; \phi(g_1 g_2) = \phi(g_1)\phi(g_2) \quad \forall g_1, g_2 \in G \]

\emph{Bijection:} a surjective and injective mapping 
\begin{enumerate}
    \item Surjective (onto) - every element is the image of an element in the domain, $\forall y\in Y, \exists x \in X: f(x) = y$
    \item Injective (one-to-one) - $f(x_1) = f(x_2) \implies x_1 = x_2$
\end{enumerate}

\emph{Isomorphism:} a bijective homomorphism 
\begin{itemize}
    \item $\mathcal C_2 \cong \mathcal S_2$
    \item $\mathcal D_3 \cong \mathcal S_3$
    \item Every group of prime order is isomorphic to $\mathcal C_p$
\end{itemize}
Isomorphic group share structural properties (both are finite, both are abelian, both share elements of the same order if one of them does)

To prove to groups are not isomorphic, show they have a structural difference.

\textbf{Subgroup:} a subset $H \subset G$ that is closed, has the identity element, and has an inverse for every element (it automatically satisfies the associative law)

To show $H$ is a subgroup, prove that $H \neq \emptyset$ and $h_1h_2^{-1} \in H \forall h_1, h_2 \in H$

\textbf{Kernel:} the kernel of a group homomorphism $\phi: G\to G'$ is 
\[\ker \phi = \{g \in G: \phi(g) = e'\}\]
\begin{itemize}
    \item $\ker(\phi)$ is a subgroup of $G$
    \item $\phi$ is injective if and only if $\ker(\phi) = \{e\}$
\end{itemize}

\textbf{Cosets:} For $H \subset G$, the coset for each $g \in G$ is $gH = \{gh: h\in H\}$
\begin{itemize}
    \item Every element of $G$ is in some coset of $H$
    \item Every coset of $H$ has the same number of elements
    \item Two cosets of $H$ are either equal or disjoint  
\end{itemize}

\textbf{Lagrange's Theorem:} the order of a subgroup divides the order of a group
\[\#G = (G:H)\#H\]
where $(G:H)$, the \emph{index of H in G}, is the number of distinct cosets of $H$. 

\emph{Corollary:} the order of an element $g \in G$ divides the order of the group 

\textbf{Product Group:} $G_1 \times G_2 = \{(a, b) : a\in G_1 \text{ and } b \in G_2\}$ with multiplication defined by 
\[(a, b) \cdot (a', b') = (a\cdot a', b\cdot b')\]

\textbf{Unproven but useful assertions of Group Theory:}
\begin{itemize}
    \item For $p$ prime and $G$ of order $p^2$, $G$ is abelian 
    \item Sylow's theorem: for $G$ finite, $p$ prime, with $p^n \bigg\vert \#G\; (n \geq 1)$, $G$ has a subgroup of order $p^n$ (but not the converse!)
    \item Structure theorem for Finite Abelian Groups: with $G$ finite abelian, there are prime powers (integers) so that 
    \[G \cong (\Z/m_1\Z) \times (\Z/m_2\Z) \times \dots \times (\Z/m_r\Z)\]
\end{itemize}

\section*{Rings}
\textbf{Ring:} set $R$ with two operations such that 
\begin{enumerate}
    \item $(R, +)$ is an abelian group with identity $0$
    \item $(R, \cdot)$ is closed, has associativity, and has an identity ($1$)
    \item $a (bc) = (ab)c \quad \forall a, b, c \in R$
\end{enumerate}

\emph{Properties:}
\begin{itemize}
    \item $0a = 0\quad \forall a \in R$
    \item $(-a)(-b) = ab \quad \forall a, b \in R$
\end{itemize}

\textbf{Ring homomorphism:} $\phi: R \to R'$ satisfying 
\begin{enumerate}
    \item $\phi(1) = 1$\\
    \item $\phi(a + b) = \phi(a) + \phi(b)$\\ 
    \item $\phi(ab) = \phi(a)\phi(b)$
\end{enumerate}

\textbf{Kernel:} $\ker(\phi) = \{a \in R: \phi(a) = 0\}$

\textbf{Important rings:}
\begin{itemize}
    \item $\Z \subset \mathbb Q \subset \R \subset \mathbb C$ 
    \item $\Z/m\Z$ (not a subring of $\mathbb C$ but with homomorphism $\phi(a) = a \mod m, \phi: \Z \to \Z/m\Z$)
    \item Gaussian integers $\Z[i] = \{a + bi: a, b\in \Z\}$
    \item Polynomial Rings $R[x] = \sum a_k x^k$
    \item Quaternions $\mathbb H = \{a + bi + cj + dk: a, b, c, d \in \R\}$ (non-commutative)
    \item $M_2(\R)$ (two-by-two matrices with real entries)
    \item $\Z/p\Z$ with prime $p$ is a field 
\end{itemize}

\textbf{Important Homomorphisms:}
\begin{itemize}
    \item Evaluation map $E_c: R[x] \to R$ with $E_c(f) = f(c)$ (kernel is polynomials with factor $x - c$)
    \item For every ring, there is a \emph{unique} homomorphism $\phi: \Z \to R$
\end{itemize}

\textbf{Field:} a commutative ring with every non-zero element having an inverse 


\textbf{Integral Domain:} a ring where $ab = 0$ implies $a = 0$ or $b = 0$

\emph{Examples:} $\R$, $\Z$, $\Z[i]$

\emph{Every field is an integral domain but not every integral domain is a field. }

\emph{Cancellation Property:} $ab = ac \iff b=c \text{ or } a =0$. A commutative ring has this property if and only if it is an integral domain.

\textbf{Unit Groups:} the subset $R^* \subset R$ where 
\[R^* = \{a \in R: \exists b \in R, ab = 1\}\] 
forms a group under multiplication.

\emph{Examples:}
\begin{itemize}
    \item $\Z^* = \{\pm 1\}$
    \item $\Z[i]^* = \{\pm 1, \pm i\}$
    \item $\R[x]^* = \R^*$
    \item $(Z/m\Z)^* = \{a \mod m: \gcd(a, m) = 1\}$
    \item $(\Z/p\Z)^* = \{1, \dots,\, p - 1\}$ (and $\Z/p\Z$ is a field)
\end{itemize}

A ring is a field if and only if $R^* = R - \{0\}$

\textbf{Product of Unit Groups:} With $R_1, \dots,\, R_n$ commutative,
\[(R_1 \times \dots \times R_n)^* \cong R_1^* \times \dots \times R_n^*\]

\textbf{Ideals:} a non-empty subset $I \subseteq R$ with 
\begin{enumerate}
    \item $a, b \in I \implies a + b \in I$
    \item $a \in I, r \in R \implies ra \in I$
\end{enumerate}

\textbf{Principal ideal:} $(c) = cR = \{rc : r\in R\}$

Every ring has at least two ideals: $(0) = 0R = \{0\}$ and $(1) = 1R = R$

\textbf{Coset:} $a \in R$, 
\[a + I = \{a + c : c \in I\}\]

Note that $a \in a + I$ because $0 \in I$ so $a + 0 \in I$. 

If $b - a \in I$ then $b \equiv a \mod I$

\textbf{Quotient:} $R/I$ is the collection of distinct cosets of $I$ and is governed by the group operations 
\begin{align*}
    (a + I) + (b + I) &= (a+b) + I\\ 
    (a + I)(b + I) &= ab + I
\end{align*}

\emph{Properties:}
\begin{enumerate}
    \item $a' + I = a + I \iff a' - a \in I$
    \item Addition and multiplication of cosets is well defined 
    \item $R/I$ is a commutative ring
\end{enumerate}

\textbf{Isomorphism Theorem:} 
\begin{enumerate}
    \item With $I \subset R$, $\phi: R \to R/I$, $\ker(\phi) = I$
    \item With $\phi: R \to R'$, $\ker(\phi) = \{a\in R: \phi(a) = 0\}$ is an ideal of R, $\phi$ is injective iff $\ker(\phi) = (0) = I$
\end{enumerate}

\section*{Homework Results}

\textbf{Generator:} If $G$ is a finite cyclic group of order $n$, and $g$ is a generator of $G$. $g^k$ is a generator of $G$ if and only if $\gcd(k, n) = 1$

\emph{Lemma:} $g\in G$ with $o(g) = n, k \geq 1$. $o(g^k) = n/\gcd(n, k)$

\textbf{Homomorphism:} If $\phi$ is a bijective homomorphism, $\phi^{-1}$ is a homomorphism 

\textbf{Center:} $Z(G) = \{g \in G: gh = hg \quad \forall h \in G\}$
\begin{itemize}
    \item $Z(G)$ is the set of elements in $G$ that commute with every other element
    \item It is a subgroup 
    \item $Z(G) = G$ when $G$ is abelian 
    \item The center of $S_n$ is trivial $n \geq 3$
    \item The center of $D_n$ is trivial for odd $n \geq 3$ and consists of identity and 180 degree rotations for even $n \geq 3$
    \item The center of $Q$ is $\{1, -1\}$
\end{itemize}

\textbf{Subgroups:} If $G$ is a finite group whose only subgroups are $\{e\}$ and $G$, then $G = \{e\}$ or it is cyclic of prime order. 

\textbf{Gaussian integers:} $\Z[\sqrt d] = \{a + b\sqrt D: a, b \in \Z\}$ If $D \geq 0$ then it is a subring of $\R$. Otherwise it is a subring of $\mathbb C$

\textbf{Fermat's Little Theorem:} with $p$ prime, $a \in \Z$ and $p \nmid a$
\[a^{p-1} \equiv 1 \mod p\]

\textbf{Ideals:}
\begin{itemize}
    \item $R$ is a field if and only if its only ideals are the zero ideal $(0)$ and the entire ring $R$
    \item every ideal in $\Z$ is a principal ideal.
\end{itemize}
\end{document}